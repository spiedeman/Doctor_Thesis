%! TEX program = xelatex

\documentclass{article}

\usepackage[paper=a4paper,left=31.7mm,right=31.7mm,top=25.4mm,bottom=25.4mm,headheight=12pt,headsep=17.5pt,footskip=10.4mm]{geometry}

\usepackage{xeCJK}
\usepackage{cjkindent}
\usepackage{amsmath}
\usepackage{amssymb}
\usepackage{cite}

\setCJKmainfont[ItalicFont=STKaitiSC-Regular]{Noto Serif CJK SC}

\begin{document}
\section{first}
  弗里德曼方程
  \begin{align}
    \frac{\dot{a}^2}{a^2}+\frac{K}{a^2} &=\frac{8\pi G}{3}\rho \\
    \frac{\ddot{a}}{a}+\frac{2\dot{a}^2}{a^2}+\frac{2k}{a^2}&=4\pi
    G{\left(\rho-p\right)}.
  \end{align}
  由$\dot{H}=\frac{\ddot{a}}{a}-\dot{H}^2$及$H=\frac{\dot{a}}{a}$得到
  \begin{align}
    \label{eq:friedmann-equation}
    H^2 &=\frac{8\pi G}{3}\rho -\frac{K}{a^2} \\
    \label{eq:accelaration-equation}
    \dot{H}+H^2 &= \frac{\ddot{a}è}{a}=-\frac{4\pi
    G}{3}{\left(\rho+3p\right)}.
  \end{align}
  以及连续性方程
  \begin{align}
    \label{eq:continuation-equation}
    \dot{\rho}=-3H{\left(\rho+p\right)}.
  \end{align}

  物态方程
  \begin{align}
    \label{eq:state-equation}
    p=w\rho  
  \end{align}
  带入连续性方程$(\ref{eq:continuation-equation})$得到能量密度$\rho$与尺度因子$a(t)$的关系
  \begin{align}
    \rho a^{3{\left(1+w\right)}}=\text{constant}
  \end{align}
  将其带入弗里德曼方程$(\ref{eq:friedmann-equation})$并令$K=0$,解出哈勃参数及其导数为
  \begin{align}
    \label{eq:hubble-parameter}
    H &=H_0 a^{-\frac{3(1+w)}{2}}\\
    \label{eq:1st-hubble-parameter}
    \dot{H} &= -\frac{3(1+w)}{2}H^2
  \end{align}
  下标$0$表示当前时刻,并取$a_0=1$。进一步得到尺度因子$a(t)$
  \begin{align}
    \label{eq:scale-factor}
    a = H_0 t^{\frac{2}{3(1+w)}} = H_0
    \begin{cases}
      t^{2 /3},\qquad & w=0 \\
      t^{1 /2},\qquad & w=1 /3
    \end{cases}
  \end{align}
  因为物态参数$w\in [-1,
  1]$,所以$\dot{H}<0$恒成立,不论是加速膨胀还是减速膨胀期。因此在宇宙整个的演化过程中,哈勃参数$H$单调减小。根据哈勃定律,退行速度$v_{tui}=Hd$。以光速$c=1$为界,相距$d=1
  /H$时,光子前进的距离正好被膨胀的效应抵消。这个临界距离被定义为哈勃半径
  \begin{align}
    \label{eq:hubble-distance}
    r_{H}=\frac{1}{H}
  \end{align}
  因而哈勃半径单调增大,未来能观测到的区域将越来越大。出于方便求解方程的目的,引入共动时间$\eta$作为新的时间变量,定义为
  \begin{align}
    \label{eq:conformal-time}
    d\eta = \frac{dt}{a}
  \end{align}
  约定$\dot{a}$和$a^\prime$分别是对$t$和$\eta$的导数。根据链式规则,两种导数按照如下规则互相转换
  \begin{align}
    \label{eq:t-to-eta}
    \frac{d}{d\eta}=a \frac{d}{dt}
  \end{align}
  在共动坐标系下,可以定义对应的共动哈勃半径
  \begin{align}
    \label{eq:comoving-hubble-distance}
    r_{\mathcal{H}} = \frac{1}{\mathcal{H}} = \frac{1}{aH} = \frac{r_{H}}{a}
  \end{align}
  这里花体表示在共动坐标系中对应的物理量,$r_{\mathcal{H}}$的行为稍有不同,
  \begin{align}
    r_{\mathcal{H}} = \frac{1}{\dot{a}} =
    \begin{cases}
      \text{单调减小},\qquad & \ddot{a}>0 \\
      \text{单调增大},\qquad &\dot{a} < 0
    \end{cases}
  \end{align}

  根据方程$(\ref{eq:1st-hubble-parameter})$,当$w\sim -1$时,$\dot{H}\sim
  0$,故$H\sim
  \text{常数}$。此时宇宙近似为德-西特宇宙,几乎以指数的速度快速膨胀。尺度因子的二阶导数$\ddot{a}$为
  \begin{align}
    \label{eq:2nd-scale-factor}
    \ddot{a}=a{\left(\dot{H}+H^2\right)}=-\frac{1+3w}{2}aH^2
  \end{align}
  当强能量条件被破坏${\left(\rho+3p\right)}<0\Leftrightarrow
  {\left(1+3w\right)}<0$,导致$\ddot{a}>0$,宇宙会加速膨胀。

\section{second}
  规范变换和规范不变量。由于物理规律不依赖于所用的坐标系,因而在任何情况下,只有不依赖于坐标系选择的物理量
  才是值得讨论的。所以当我们研究宇宙中物质扰动的演化规律时,目标就是要找到非坐标依赖的扰动量。首先定义\textbf{规范变换}为物理量在无穷小坐标变换下的变换规则,而\textbf{规范不变量}为在规范变换下不变的物理量。
  考虑无穷小坐标变换
  \begin{align}
    \label{eq:coordinate-transformation}
    x^{\alpha}\rightarrow \tilde{x}^{\alpha}=x^{\alpha} + \xi^{\alpha}.
  \end{align}
  其中$\xi^{\alpha}$为无穷小时空的函数。在该变换下,四-标量(矢量、张量)的扰动分别遵循如下的变换规则
  \begin{align}
    \label{eq:gauge-transformation-scalar}
    \delta{q} \rightarrow \tilde{q} &=\delta{q}-^{(0)}q_{\alpha}\xi^{\alpha} \\
    \label{eq:gauge-transformation-vector}
    \delta{u}_{\alpha}\rightarrow \delta{\tilde{u}}_{\alpha}&=
    \delta{u}_{\alpha}-^{(0)}u_{\alpha,\gamma}\xi^{\gamma}-^{(0)}u_{\gamma}\xi^{\gamma}_{\
    ,\alpha} \\
    \label{eq:gauge-transformation-tensor}
    \delta{g_{\alpha\beta}}\rightarrow \delta{\tilde{g}_{\alpha\beta}} &=
    \delta{g_{\alpha\beta}}-^{(0)}g_{\alpha\beta,\gamma}\xi^{\gamma}-^{(0)}g_{\gamma\beta}\xi^{\gamma}_{\
    ,\alpha}-^{(0)}g_{\alpha\gamma}\xi^{\gamma}_{\ ,\beta}.
  \end{align}
  
  当度规扰动为标量扰动时,扰动度规形式为
  \begin{align}
    \label{eq:scalar-perturbation-metric}
    ds^2=a^2{\left[(1+2\phi)d\eta^2+2B_{,i}dx^{i}d\eta-{\left((1-2\psi)\delta_{ij}-2E_{,ij}\right)}dx^{i}dx^{j}\right]}.
  \end{align}
  其中$\phi, \psi, B,
  E$为四个标量函数,根据张量扰动的规范变换可以得到它们的变换规则,再通过线性组合能够得到两个规范不变量
  \begin{equation}
    \begin{split}
      \label{eq:gauge-invariant-varible}
      \Phi &\equiv\phi-\frac{1}{a}{\left[a{\left(B-E^\prime\right)}\right]}^{\prime}\\
      \Psi &\equiv \psi+ \frac{a^\prime}{a}{\left(B-E^\prime\right)}.
    \end{split}
  \end{equation}
  
  规范不变量的存在表明四个标量函数中只有两个独立函数。通过人为引入两个约束条件,可以消去多余的非物理自由度。
  不同的引入约束条件的方式代表不同的规范,效果等价于选取一个唯一的坐标系或一类坐标簇。常用的规范有\textbf{纵向规范}和\textbf{同步规范}。

  \textbf{纵向规范}:$B=E=0$。度规的扰动形式由$\phi$、$\psi$两个标量函数刻画,
  \begin{align}
    \label{eq:conformal-newtonian}
    ds^2=a^2{\left[(1+2\phi)d\eta^2-(1-2\psi)\delta_{ij}dx^{i}dx^{j}\right]}.
  \end{align}
  从$(\ref{eq:gauge-invariant-varible})$可知在纵向规范下,规范不变量就是度规的扰动,特别有$\Psi=\psi
  $,$\Phi=\phi$。

  \textbf{同步规范}:$\phi=B=0$。度规的扰动形式由$\psi$、$E$两个标量函数刻画,
  \begin{align}
    \label{eq:syncronous-gauge}
    ds^2=a^2{\left[-{\left((1-2\psi)\delta_{ij}-2E_{,ij}\right)}dx^{i}dx^{j}\right]}.
  \end{align}
  从$(\ref{eq:gauge-invariant-varible})$可知在同步规范下,规范不变量与扰动的关系为
  \begin{align}
    \Phi=\frac{1}{a}{\left[aE^\prime\right]}^{\prime},\qquad
    \Psi=\psi-\frac{a^\prime}{a}E^\prime.
  \end{align}
  
  \subsection{扰动方程}
  为了获知规范不变量如何随时间演化,首先要得到规范不变量满足的方程。由爱因斯坦方程很容易得到这个方程为
  \begin{align}
    \label{eq:gauge-invariant-perturbation-equation}
    \overline{\delta G}^{\alpha}_{\beta}= 8\pi G\overline{\delta
    T}^{\alpha}_{\ \beta}. 
  \end{align}
  规范不变量$\overline{\delta T}^{\alpha}_{\
  \beta}$可以分解为标量、矢量、张量三个互不影响的部分。背景为FRW度规的情况下,直接计算规范不变的张量扰动$\overline{\delta
G}^{\alpha}_{\ \beta}$可以得到三种扰动分别满足的方程组
  
  \textbf{标量扰动}:
  \begin{align}
    \Delta\Phi-3\mathcal{H}{\left(\Phi^\prime+\mathcal{H}\Psi\right)} =
    4\pi Ga^2\overline{\delta T}^{0}_{\ 0}&, 
    \label{eq:equation-gauge-scalar-perturbation1}\\
    {\left(\Phi^\prime+\mathcal{H}\Phi\right)}_{,i}=4\pi
    Ga^2\overline{\delta T}^{0}_{\ i}&,
    \label{eq:equation-gauge-scalar-perturbation2}\\
    \begin{split}
      {\left[\Phi^{\prime\prime}+\mathcal{H}{\left(2\Psi+\Phi\right)}^{\prime}+(2\mathcal{H^\prime}+\mathcal{H}^2)\Phi 
  +\frac{1}{2}\Delta(\Phi-\Psi)\right]}&\delta_{ij} \\
      -\frac{1}{2}{\left((\Phi-\Psi)\right)}_{,ij} = -4\pi
  Ga^2\overline{\delta T}^{i}_{\ j}&.
    \end{split}
    \label{eq:equation-gauge-scalar-perturbation3}
  \end{align}
  
  \textbf{矢量扰动}:
  \begin{align}
    \label{eq:equation-gauge-vector-perturbation}
    \Delta\overline{V}_{i}=16\pi Ga^2\overline{\delta T}^{0}_{\ i(V)}, \\  
    {\left(\overline{V}_{i,j}+\overline{V}_{j,i}\right)}^{\prime} + 
    2\mathcal{H}{\left(\overline{V}_{i,j}+\overline{V}_{j,i}\right)} = 
    -16\pi Ga^2&\overline{\delta T}^{i}_{\ j(V)}.
  \end{align}
  
  \textbf{张量扰动}:
  \begin{align}
    \label{eq:equation-gauge-tensor-perturbation}
    h^{\prime\prime}_{ij}+2\mathcal{H}h^\prime_{ij}-\Delta h_{ij}=16\pi
    Ga^2\overline{\delta T}^{i}_{\ j(T)}.
  \end{align}
  
  这些方程的成立不依赖于某个特定的坐标系。然而由于在纵向规范下,有特殊关系$\Phi=\phi$和$\Psi=\psi$,使我们
  能够在该规范下求得扰动满足的各种方程,再做替换便可直接得到规范不变量所满足的方程。

  \subsection{完美流体}
  得到扰动方程后,以完美流体为例,求解规范不变量的演化。完美流体的能动量张量为
  \begin{align}
    \label{eq:perfect-fluid}
    T^{\alpha}_{\beta}=(\varepsilon+p)u^{\alpha}u_{\beta}-p\delta^{\alpha}_{\ \beta}.
  \end{align}
  相应的规范不变量为
  \begin{align}
    \label{eq:perfect-fluid-gauge-invariant-variable}
    \overline{\delta{T}}^{0}_{\ 0} &= \overline{\delta{\varepsilon}} \\
    \overline{\delta{T}}^{0}_{\ i} &=
    \frac{1}{a}(\varepsilon_0+p_0){\left(\overline{\delta{u}}_{\shortparallel
          i}+\overline{\delta{u}}_{\perp i}\right)} \\
    \overline{\delta{T}}^{i}_{\ j}&= -\overline{\delta{p}}\delta^{i}_{\ j}.
  \end{align}
  $\overline{\delta{\varepsilon}}$、$\overline{\delta{u}}_{\shortparallel
  i}$、$\overline{\delta{p}}$分别是能量密度扰动、无旋速度扰动、压强扰动对应的规范不变量。
  标量扰动的贡献只来自于这几项,正比于速度扰动的无源分量$\overline{\delta{u}}_{\perp
  i}$的项只对矢量扰动产生贡献。\\
  当$i\neq
  j$时,$\delta{T}^{i}_{j}=0$,方程$(\ref{eq:equation-gauge-scalar-perturbation3})$约化为
  \begin{align}
    {\left(\Phi-\Psi\right)}_{,ij}=0\qquad (i\neq j).
  \end{align}
  有唯一解$\Phi=\Psi$。从而标量扰动满足的方程组进一步被化简为
  \begin{align}
    \label{eq:simplified-equation-gauge-scalar-perturbation}
    \Delta \Phi -3\mathcal{H}{\left(\Phi^\prime+\mathcal{H}\Phi\right)} =
    4\pi Ga^2\overline{\delta{\varepsilon}}, \\
    {\left(a\Phi\right)}^{\prime}_{,i}=4\pi
    Ga^2{\left(\varepsilon_0+p_0\right)}\overline{\delta{u}}_{\shortparallel i}, \\
    \Phi^{\prime\prime}+3\mathcal{H}\Phi^\prime+(2\mathcal{H}^{\prime}+\mathcal{H}^2)\Phi=
    4\pi Ga^2\overline{\delta{p}}.
  \end{align}
  
  
\end{document}

