
%! TEX program = xelatex

\documentclass{article}

\usepackage[paper=a4paper,left=31.7mm,right=31.7mm,top=25.4mm,bottom=25.4mm,headheight=12pt,headsep=17.5pt,footskip=10.4mm]{geometry}

\usepackage{xeCJK}
\usepackage{cjkindent}
\usepackage{amsmath}
\usepackage{amssymb}
\usepackage{cite}

\setCJKmainfont[ItalicFont=STKaitiSC-Regular]{Noto Serif CJK SC}

\begin{document}
弗里德曼方程
\begin{align}
  \frac{\dot{a}^2}{a^2}+\frac{K}{a^2} &=\frac{8\pi G}{3}\rho \\
  \frac{\ddot{a}}{a}+\frac{2\dot{a}^2}{a^2}+\frac{2k}{a^2}&=4\pi
  G{\left(\rho-p\right)}.
\end{align}
由$\dot{H}=\frac{\ddot{a}}{a}-\dot{H}^2$及$H=\frac{\dot{a}}{a}$得到
\begin{align}
  \label{eq:friedmann-equation}
  H^2 &=\frac{8\pi G}{3}\rho -\frac{K}{a^2} \\
  \label{eq:accelaration-equation}
  \dot{H}+H^2 &= \frac{\ddot{a}è}{a}=-\frac{4\pi
  G}{3}{\left(\rho+3p\right)}.
\end{align}
以及连续性方程
\begin{align}
  \label{eq:continuation-equation}
  \dot{\rho}=-3H{\left(\rho+p\right)}.
\end{align}

物态方程
\begin{align}
  \label{eq:state-equation}
  p=w\rho  
\end{align}
带入连续性方程$(\ref{eq:continuation-equation})$得到能量密度$\rho$与尺度因子$a(t)$的关系
\begin{align}
  \rho a^{3{\left(1+w\right)}}=\text{constant}
\end{align}
将其带入弗里德曼方程$(\ref{eq:friedmann-equation})$并令$K=0$,解出哈勃参数及其导数为
\begin{align}
  \label{eq:hubble-parameter}
  H &=H_0 a^{-\frac{3(1+w)}{2}}\\
  \label{eq:1st-hubble-parameter}
  \dot{H} &= -\frac{3(1+w)}{2}H^2
\end{align}
下标$0$表示当前时刻,并取$a_0=1$。进一步得到尺度因子$a(t)$
\begin{align}
  \label{eq:scale-factor}
  a = H_0 t^{\frac{2}{3(1+w)}} = H_0
  \begin{cases}
    t^{2 /3},\qquad & w=0 \\
    t^{1 /2},\qquad & w=1 /3
  \end{cases}
\end{align}
因为物态参数$w\in [-1,
1]$,所以$\dot{H}<0$恒成立,不论是加速膨胀还是减速膨胀期。因此在宇宙整个的演化过程中,哈勃参数$H$单调减小。根据哈勃定律,退行速度$v_{tui}=Hd$。以光速$c=1$为界,相距$d=1
/H$时,光子前进的距离正好被膨胀的效应抵消。这个临界距离被定义为哈勃半径
\begin{align}
  \label{eq:hubble-distance}
  r_{H}=\frac{1}{H}
\end{align}
因而哈勃半径单调增大,未来能观测到的区域将越来越大。出于方便求解方程的目的,引入共动时间$\eta$作为新的时间变量,定义为
\begin{align}
  \label{eq:conformal-time}
  d\eta = \frac{dt}{a}
\end{align}
约定$\dot{a}$和$a^\prime$分别是对$t$和$\eta$的导数。根据链式规则,两种导数按照如下规则互相转换
\begin{align}
  \label{eq:t-to-eta}
  \frac{d}{d\eta}=a \frac{d}{dt}
\end{align}
在共动坐标系下,可以定义对应的共动哈勃半径
\begin{align}
  \label{eq:comoving-hubble-distance}
  r_{\mathcal{H}} = \frac{1}{\mathcal{H}} = \frac{1}{aH} = \frac{r_{H}}{a}
\end{align}
这里花体表示在共动坐标系中对应的物理量,$r_{\mathcal{H}}$的行为稍有不同,
\begin{align}
  r_{\mathcal{H}} = \frac{1}{\dot{a}} =
  \begin{cases}
    \text{单调减小},\qquad & \ddot{a}>0 \\
    \text{单调增大},\qquad &\dot{a} < 0
  \end{cases}
\end{align}

根据方程$(\ref{eq:1st-hubble-parameter})$,当$w\sim -1$时,$\dot{H}\sim
0$,故$H\sim
\text{常数}$。此时宇宙近似为德-西特宇宙,几乎以指数的速度快速膨胀。尺度因子的二阶导数$\ddot{a}$为
\begin{align}
  \label{eq:2nd-scale-factor}
  \ddot{a}=a{\left(\dot{H}+H^2\right)}=-\frac{1+3w}{2}aH^2
\end{align}
当强能量条件被破坏${\left(\rho+3p\right)}<0\Leftrightarrow
{\left(1+3w\right)}<0$,导致$\ddot{a}>0$,宇宙会加速膨胀。

\end{document}

