%---------------------------------------------------------------------------%
%->> 封面信息及生成
%---------------------------------------------------------------------------%
%-
%-> 中文封面信息
%-
\confidential{}% 密级:只有涉密论文才填写
\schoollogo{scale=0.095}{ucas_logo}% 校徽
\title{拐点暴胀模型}% 论文中文题目
\author{徐武涛}% 论文作者
\advisor{郭宗宽~研究员~中国科学院理论物理研究所}% 指导教师:姓名 专业技术职务 工作单位
\advisors{}% 指导老师附加信息 或 第二指导老师信息
\degree{博士}% 学位:学士、硕士、博士
\degreetype{理学}% 学位类别:理学、工学、工程、医学等
\major{理论物理}% 二级学科专业名称
\institute{中国科学院理论物理研究所}% 院系名称
\date{2019~年~12~月}% 毕业日期:夏季为6月、冬季为12月
%-
%-> 英文封面信息
%-
\TITLE{Inflection Point Inflation}% 论文英文题目
\AUTHOR{Xu Wutao}% 论文作者
\ADVISOR{Supervisor: Professor Guo Zongkuan}% 指导教师
\DEGREE{Doctor}% 学位:Bachelor, Master, Doctor。封面格式将根据英文学位名称自动切换,请确保拼写准确无误
\DEGREETYPE{Philosophy}% 学位类别:Philosophy, Natural Science, Engineering, Economics, Agriculture 等
\MAJOR{Theoretical Physics}% 二级学科专业名称
\INSTITUTE{Institute of Theoretical Physics, Chinese Academy of Sciences}% 院系名称
\DATE{December, 2019}% 毕业日期:夏季为June、冬季为December
%-
%-> 生成封面
%-
\maketitle% 生成中文封面
\MAKETITLE% 生成英文封面
%-
%-> 作者声明
%-
\makedeclaration% 生成声明页
%-
%-> 中文摘要
%-
\intobmk\chapter*{摘\quad 要}% 显示在书签但不显示在目录
\setcounter{page}{1}% 开始页码
\pagenumbering{Roman}% 页码符号


本文首先介绍了现代宇宙学的基本理论。包含标准宇宙学模型中的背景演化,以及遭遇到的
一些疑难问题。然后是以此为背景发展出的暴胀宇宙学,介绍了基本的理论模型。接着着重
介绍了工作相关的规范扰动理论,包括三种扰动类型、各自满足的规范不变方程组。并以
标量扰动为例在某些情况下推导了扰动方程,并讨论了解的性质。以标量扰动和张量扰动为例
介绍了相关的功率谱的计算。


\keywords{宇宙微波背景辐射,原初黑洞,二级引力波,慢滚参数}% 中文关键词
%-
%-> 英文摘要
%-
\intobmk\chapter*{Abstract}% 显示在书签但不显示在目录

write lately.

\KEYWORDS{cosmic microwave background, primodial black hole, second order
gravitional wave, slow-roll parameters}% 英文关键词
%---------------------------------------------------------------------------%
