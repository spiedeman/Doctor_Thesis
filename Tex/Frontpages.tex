%---------------------------------------------------------------------------%
%->> 封面信息及生成
%---------------------------------------------------------------------------%
%-
%-> 中文封面信息
%-
\confidential{}% 密级:只有涉密论文才填写
\schoollogo{scale=0.095}{ucas_logo}% 校徽
\title{拐点暴涨模型研究}% 论文中文题目
\author{徐武涛}% 论文作者
\advisor{郭宗宽~研究员}% 指导教师:姓名 专业技术职务 工作单位
\degree{博士}% 学位:学士、硕士、博士
\degreetype{理学}% 学位类别:理学、工学、工程、医学等
\major{理论物理}% 二级学科专业名称
\institute{中国科学院理论物理研究所}% 院系名称
\date{2020~年~6~月}% 毕业日期:夏季为6月、冬季为12月
%-
%-> 英文封面信息
%-
\TITLE{Study on Inflection-Point Inflationary Models}% 论文英文题目
\AUTHOR{Xu Wutao}% 论文作者
\ADVISOR{Supervisor: Professor Guo Zongkuan}% 指导教师
\DEGREE{Doctor}% 学位:Bachelor, Master, Doctor。封面格式将根据英文学位名称自动切换,请确保拼写准确无误
\DEGREETYPE{Philosophy}% 学位类别:Philosophy, Natural Science, Engineering, Economics, Agriculture 等
\MAJOR{Theoretical Physics}% 二级学科专业名称
\INSTITUTE{Institute of Theoretical Physics, Chinese Academy of Sciences}% 院系名称
\DATE{June, 2020}% 毕业日期:夏季为June、冬季为December
%-
%-> 生成封面
%-
\maketitle% 生成中文封面
\MAKETITLE% 生成英文封面
%-
%-> 作者声明
%-
\makedeclaration% 生成声明页
%-
%-> 中文摘要
%-
\intobmk\chapter*{摘\quad 要}% 显示在书签但不显示在目录
\setcounter{page}{1}% 开始页码
\pagenumbering{Roman}% 页码符号


本文首先介绍了宇宙演化的历史与暴涨宇宙学,重点介绍了宇宙
扰动理论,给出了标量扰动和张量扰动满足的演化方程,计算了相应的原初功率谱。

在热大爆炸模型中存在一些疑难,譬如视界问题、平坦性问题、磁单极问题等,暴涨
模型认为早期宇宙存在一个近指数加速膨胀的过程,能够解决这些问题。
尽管有很多暴涨模型被提出,却仍然缺乏对暴涨场本质的认识,因此如何在基本理论(譬如,超引力)中构造
暴涨模型是一个有意义的研究课题。
我们在超引力中构造了拐点暴涨模型,该模型预言的标量扰动和张量扰动的原初功率谱与宇宙微波背景辐射
观测数据一致。

另外,在超引力中也可以实现双拐点暴涨模型,其中一个拐点可以产生与宇宙微波背景辐射观测一致的原初功率谱,
另一个拐点使得标量扰动功率谱在小尺度上产生一个峰值。小尺度上扰动的增强能够
导致原初黑洞在辐射为主时期通过引力塌缩产生,同时扰动也会诱导随机引力波背景的产生。
我们数值计算了引力波的能谱,发现该引力波信号有望被未来的LISA和Taiji实验所观测到。

\keywords{宇宙学,拐点暴涨,原初黑洞,诱导引力波,扰动理论}% 中文关键词
%-
%-> 英文摘要
%-
\intobmk\chapter*{Abstract}% 显示在书签但不显示在目录

First of all, we review the evolution of
the Universe and inflationary cosmology. We pay attention to the
cosmological perturbation theory, derive the evolution equations of scalar
and tensor perturbations, and caculate the corresponding primordial
power spectra.
% First of all, we introduce modern cosmology, including
% the background evolution equation in big bang theory and some problems such as the 
% flatness problem, horizon problem, magnetic monopoles problem. Then we breifly
% introduce inflationary models. We pay attention
% to the cosmological perturbation theory, derive the evolution
% equations of scalar, vector, tensor perturbation, and caculate the
% corresponding primordial power spectra.
 

There is a short exponentially accelerated expansion in the early 
Universe in the inflationary scenario, which is proposed to solve some
problems in the big bang, such as the horizon problem, flatness
problem and monopole problem. Although a lot of inflationary models have been 
proposed, the nature of the inflation field is an open question.  
Constructing successful inflationary models in the supergravity 
framework is a valuable attempt. We constructe an inflection-point
inflationary model in supergravity, which predicts the primordial power spectra of scalar and tensor perturbations
consistent with CMB observations.

Moreover, a double-inflection-point
inflationary model is constructed in supergravity. One inflection point generates the power 
spectra consistent with CMB observations, the other generates a large peak 
in the power spectrum of scalar perturbations at small scales. The
enhancement of the power spectrum at small scales leads to the production of
primordial black holes via gravitational collapse in the radiation-dominated
era and induces a stochastic background of gravitational waves.
We numerically calculate the energy spectrum of gravitational waves and find that the signal is
expected to be observed by LISA and Taiji in the future.


\KEYWORDS{cosmology, inflection-point inflation, primordial black hole,
induced gravitional waves, perturbation theory}% 英文关键词
%---------------------------------------------------------------------------%
