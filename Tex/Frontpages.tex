%---------------------------------------------------------------------------%
%->> 封面信息及生成
%---------------------------------------------------------------------------%
%-
%-> 中文封面信息
%-
\confidential{}% 密级:只有涉密论文才填写
\schoollogo{scale=0.095}{ucas_logo}% 校徽
\title{拐点暴胀模型}% 论文中文题目
\author{徐武涛}% 论文作者
\advisor{郭宗宽~研究员~中国科学院理论物理研究所}% 指导教师:姓名 专业技术职务 工作单位
\advisors{}% 指导老师附加信息 或 第二指导老师信息
\degree{博士}% 学位:学士、硕士、博士
\degreetype{理学}% 学位类别:理学、工学、工程、医学等
\major{理论物理}% 二级学科专业名称
\institute{中国科学院理论物理研究所}% 院系名称
\date{2019~年~12~月}% 毕业日期:夏季为6月、冬季为12月
%-
%-> 英文封面信息
%-
\TITLE{Inflection Point Inflation}% 论文英文题目
\AUTHOR{Xu Wutao}% 论文作者
\ADVISOR{Supervisor: Professor Guo Zongkuan}% 指导教师
\DEGREE{Doctor}% 学位:Bachelor, Master, Doctor。封面格式将根据英文学位名称自动切换,请确保拼写准确无误
\DEGREETYPE{Philosophy}% 学位类别:Philosophy, Natural Science, Engineering, Economics, Agriculture 等
\MAJOR{Theoretical Physics}% 二级学科专业名称
\INSTITUTE{Institute of Theoretical Physics, Chinese Academy of Sciences}% 院系名称
\DATE{December, 2019}% 毕业日期:夏季为June、冬季为December
%-
%-> 生成封面
%-
\maketitle% 生成中文封面
\MAKETITLE% 生成英文封面
%-
%-> 作者声明
%-
\makedeclaration% 生成声明页
%-
%-> 中文摘要
%-
\intobmk\chapter*{摘\quad 要}% 显示在书签但不显示在目录
\setcounter{page}{1}% 开始页码
\pagenumbering{Roman}% 页码符号


本文首先介绍了现代宇宙学的基本理论。包含标准宇宙学模型中的背景演化,以及遭遇到的
一些疑难问题。然后是以此为背景发展出的暴胀宇宙学,介绍了基本的理论模型。接着着重
介绍了工作相关的规范扰动理论,包括三种扰动类型、各自满足的规范不变方程组。并以
标量扰动为例在某些情况下推导了扰动方程,并讨论了解的性质。以标量扰动和张量扰动为例
介绍了相关的功率谱的计算。

暴胀模型最初被提出仅仅是从唯像的角度认为早期宇宙存在一个指数加速膨胀的短暂过程
,为了解决大爆炸模型遇到的一些疑难。为了赋予暴胀理论更多的物理背景,其中一个做
法是在超引力的框架内构造可行的暴胀模型。第一个工作就是在超引力框架内构造了一个
单拐点暴胀模型,拐点的存在使得能够产生符合观测数据的早期的CMB谱。另外通过调节
参数,能够给出与观测相符的谱指标$n_{s}$和张标比$r$。

接着第二个工作中进一步在超引力中构造了双拐点暴胀模型。其中一个拐点用于产生早期
的CMB谱,另一个拐点使得标量扰动谱在小尺度上产生一个峰值。小尺度上扰动的增强能够
导致原初黑洞在辐射为主时期通过引力塌缩产生,因此原初黑洞的丰度和标量扰动诱导
产生的二阶引力波之间存在互相制约的关系。

\keywords{宇宙学,拐点暴胀,原初黑洞,二级引力波,规范扰动理论}% 中文关键词
%-
%-> 英文摘要
%-
\intobmk\chapter*{Abstract}% 显示在书签但不显示在目录

In the first part of the paper, I introduce the theory of modern cosmology,
including the background equation and some problems like flatness problem, 
horizon problem and so on. After that I breifly introduce the theory of 
inflation that aimed at solving the problems mentioned before. Then I
described the guage invariant perturbation theory including three kind of
perturbations and corresponding satisfied guage invariant equations. As an
example, I derive the equation of scalar perturbation and its solution
under some condition. At last, I show the caculation of the power spectrum
for scalar and tensor perturbations.

The theory of inflation was proposed to solve the problems incountered by
the Big Bang theory with an exponentially accelerated period at early
universe. Constructing inflation model in super gravity is one of the
practicable methods to make inflation a more theoretical theory.
My first work is to construct an inflection point inflation model in super gravity
framework. The existence of the inflection point produces the early CMB
power spectrum in agreement with the experiments like WMAP and Planck. In
addition, the model predicts the reasonable index of spectrum $ns$ and
tensor-to-scalar ratio $r$ by fine tuning the parameters.

The second work followed by constructing a double inflection points
inflation in super gravity framework too. Here we have two inflection
points. One is used to generate the power spectra consistent with the CMB
constraints at large scale. The other point can generate a large peak in
the power spectrum of scalar perturbations at small scales, which leads to 
the production of primordial black holes via gravititional collapse in the 
radiation-dominated era. Therefor, the induced gravititional waves are used
to constain on the abundance of primodrdial black holes and vice versa.

\KEYWORDS{cosmology, inflection point inflation, primordial black hole, second order
gravitional wave, guage invariant perturbation}% 英文关键词
%---------------------------------------------------------------------------%
