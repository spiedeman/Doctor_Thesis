%---------------------------------------------------------------------------%
%->> 封面信息及生成
%---------------------------------------------------------------------------%
%-
%-> 中文封面信息
%-
\confidential{}% 密级:只有涉密论文才填写
\schoollogo{scale=0.095}{ucas_logo}% 校徽
\title{拐点暴胀模型研究}% 论文中文题目
\author{徐武涛}% 论文作者
\advisor{郭宗宽~研究员}% 指导教师:姓名 专业技术职务 工作单位
\degree{博士}% 学位:学士、硕士、博士
\degreetype{理学}% 学位类别:理学、工学、工程、医学等
\major{理论物理}% 二级学科专业名称
\institute{中国科学院理论物理研究所}% 院系名称
\date{2020~年~6~月}% 毕业日期:夏季为6月、冬季为12月
%-
%-> 英文封面信息
%-
\TITLE{Study On Inflection-Point Inflation Models}% 论文英文题目
\AUTHOR{Xu Wutao}% 论文作者
\ADVISOR{Supervisor: Professor Guo Zongkuan}% 指导教师
\DEGREE{Doctor}% 学位:Bachelor, Master, Doctor。封面格式将根据英文学位名称自动切换,请确保拼写准确无误
\DEGREETYPE{Philosophy}% 学位类别:Philosophy, Natural Science, Engineering, Economics, Agriculture 等
\MAJOR{Theoretical Physics}% 二级学科专业名称
\INSTITUTE{Institute of Theoretical Physics, Chinese Academy of Sciences}% 院系名称
\DATE{June, 2020}% 毕业日期:夏季为June、冬季为December
%-
%-> 生成封面
%-
\maketitle% 生成中文封面
\MAKETITLE% 生成英文封面
%-
%-> 作者声明
%-
\makedeclaration% 生成声明页
%-
%-> 中文摘要
%-
\intobmk\chapter*{摘\quad 要}% 显示在书签但不显示在目录
\setcounter{page}{1}% 开始页码
\pagenumbering{Roman}% 页码符号


本文首先介绍了现代宇宙学。包含大爆炸宇宙学模型中的背景演化方程,以及存在的
一些疑难问题,例如平坦性疑难、视界疑难、磁单极子疑难等。
然后介绍了暴胀模型。
我们重点介绍了宇宙线性扰动理论,给出扰动满足的演化方程,计算了相应的原初
功率谱。

暴胀模型认为早期宇宙存在一个指数加速膨胀的短暂过程
,能够解决大爆炸模型中存在的一些疑难。然而尽管有很多暴胀模型被提出,却仍然
缺乏对暴胀场本质的认识,因此如何在超引力中构造暴胀模型是一个值得研究的课题。
我们的第一个工作是在超引力中构造了一个单拐点暴胀模型,该模型预言的功率谱与CMB
观测数据一致。

在第一个工作的基础上,我们又在超引力中构造了双拐点暴胀模型。
其中一个拐点可以产生与CMB观测一致的功率谱,
另一个拐点使得标量扰动谱在小尺度上产生一个峰值。小尺度上扰动的增强能够
导致原初黑洞在辐射为主时期通过引力塌缩产生,同时扰动也会诱导产生随机引力波背景。
该引力波信号有望被未来的LISA和Taiji实验所观测到。

\keywords{宇宙学,拐点暴胀,原初黑洞,诱导引力波,扰动理论}% 中文关键词
%-
%-> 英文摘要
%-
\intobmk\chapter*{Abstract}% 显示在书签但不显示在目录

First of all, we introduce the modern cosmology in this paper, including
the background evolution equation in big bang theory and some problems such as the 
flatness problem, the horizon problem, the magnetic monopoles problem and
so on. Then we breifly introduce the inflation model. And we pay attentions
to the cosmological linear perturbation theory, not only the evolution
equations for the perturbation, but also the corresponding primordial power
spectrum.

There is a short exponentially accelerated period in early universe
according to the inflation model, which is proposed to address a series of
questions. Although a lot of inflation models have been proposed, there is
still lacking of the knowledge of the essential of inflation. That's why
constructing a successful inflation model in super gravity framework is
valuable work. Our first work is to construct a single-inflection-point
inflation model in super gravity, which predicts a power spectrum
consistent with the CMB observations.

Based on the first work, we again construct a double-inflection-point
inflation model. One inflection point generates the power spectrum
consistent with them CMB observations, the other one generates a large peak 
in the power spectrum of scalar perturbation in small scales. Then
enhancement of power spectrum at small scales leads to the production of
primordial black holes via gravitational collapse in radiation-dominated
era and induces the stochastic background of gravitational waves which
could be observed by LISA and Taiji in the future.

% In the first part of the paper, I introduce the theory of modern cosmology,
% including the background equation and some problems like flatness problem, 
% horizon problem and so on. after that i breifly introduce the theory of 
% inflation that aimed at solving the problems mentioned before. then i
% described the guage invariant perturbation theory including three kind of
% perturbations and corresponding satisfied guage invariant equations. As an
% example, I derive the equation of scalar perturbation and its solution
% under some condition. At last, I show the caculation of the power spectrum
% for scalar and tensor perturbations.
% 
% The theory of inflation was proposed to solve the problems incountered by
% the Big Bang theory with an exponentially accelerated period at early
% universe. Constructing inflation model in super gravity is one of the
% practicable methods to make inflation a more theoretical theory.
% My first work is to construct an inflection point inflation model in super gravity
% framework. The existence of the inflection point produces the early CMB
% power spectrum in agreement with the experiments like WMAP and Planck. In
% addition, the model predicts the reasonable index of spectrum $ns$ and
% tensor-to-scalar ratio $r$ by fine tuning the parameters.
% 
% The second work followed by constructing a double-inflection-point
% inflation in super gravity framework too. Here we have two inflection
% points. One is used to generate the power spectra consistent with the CMB
% constraints at large scale. The other point can generate a large peak in
% the power spectrum of scalar perturbations at small scales, which leads to 
% the production of primordial black holes via gravititional collapse in the 
% radiation-dominated era. Therefor, the induced gravititional waves are used
% to constain on the abundance of primodrdial black holes and vice versa.

\KEYWORDS{cosmology, inflection-point inflation, primordial black hole,
induced gravitional wave, perturbation theory}% 英文关键词
%---------------------------------------------------------------------------%
