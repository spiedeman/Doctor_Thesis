\chapter{总结与展望}%
\label{chap:summary}
本文的主要工作为在超引力框架内构造拐点暴胀模型,并将其与诱导引力波
相联系,为限制原初黑洞丰度提供理论依据。

第一部分介绍了宇宙学背景,以及暴胀理论和宇宙学扰动理论的研究背景和现状。第二
部分介绍了现代宇宙学的基本内容。首先简单介绍了标准宇宙学模型,接着介绍了暴胀
理论提出的背景以及基本概念。之后详细介绍了规范扰动理论的知识,包括扰动的分解、
规范不变量的构造、常用的规范。以理想流体为例,给出了标量扰动分别在长波、短波
近似下的扰动方程和相应的解。之后以标量扰动和张量扰动为例详细讨论了扰动的
功率谱。第三部分涉及第一个主要工作,在超引力框架内构造符合观测数据的暴胀模型。
首先简单回顾了暴胀理论在超引力框架内的发展历程,接着讨论了跑动动能项暴胀模型。
然后是工作中的主要内容,在超引力中构造了一个带有跑动动能项的单拐点暴胀
模型。详细讨论了各种参数下,模型预言的谱指数$ns$和张标比$r$与实验结果的对比。
第四部分涉及第二个主要工作。仍然是在超引力框架内构造暴胀模型。这次构造的模型
区别在于多了一个拐点,主要用于使标量扰动的功率谱在小尺度上产生一个峰。这个峰
的存在能够使扰动在辐射为主时期重新进入视界内时由于引力塌缩的缘故生成原初黑洞。
不过这里重点在于计算标量扰动诱导产生的引力波的信号大小。标量谱在小尺度上的
增强使得诱导引力波的能量谱也产生一个峰,从而不能随意忽略诱导引力波的贡献。
通过与LISA和太极的灵敏度曲线比较,发现理论上诱导引力波存在被LISA和太极观测所证实
的可能性。

在未来的研究中,将继续探索双拐点暴胀模型的构造,诱导引力波能谱的计算。加入对原初黑洞的讨论之后,
可以对诱导引力波能谱进行相应的限制,从而对暴胀模型产生相应的限制条件。

科学的道路无止境,期待宇宙学在不久的将来能够取得突破性的进展,将人类对宇宙的认识
提升到一个新的高度。

