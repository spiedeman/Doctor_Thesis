\chapter{总结与展望}%
\label{chap:summary}
本文研究了拐点暴涨模型,主要包括在超引力中的构建和诱导引力波能谱的计算。

第一部分简要介绍了现代宇宙学的发展和现状。
第二部分介绍了暴涨理论。回顾了暴涨提出的背景以及基本思想,之后详细介绍了规范扰动理论,包括扰动的分解、
规范不变量的构造、常用的规范,推导了标量扰动和张量扰动的演化方程,在慢滚条件下求解了扰动方程,
计算了标量扰动和张量扰动的原初功率谱。
第三部分研究了拐点暴涨模型。
首先简单回顾了暴涨模型在超引力框架内的实现,接着讨论了跑动动能项暴涨模型。
我们在超引力中构造了一个带有跑动动能项的拐点暴涨
模型。详细讨论了各种参数下,模型预言的标量扰动谱指数$n_s$和张标比$r$与实验结果的对比。
第四部分研究了双拐点暴涨模型。双拐点暴涨模型预言的标量扰动在大尺度上与CMB观测一致,
在小尺度上产生一个峰。
这个峰的存在能够使扰动在辐射为主时期重新进入视界内时由于引力塌缩的缘故生成原初黑洞,
同时,标量扰动诱导随机引力波背景的产生。我们计算了引力波的能谱,发现标量谱在小尺度上的
增强使得诱导引力波的能谱也产生一个峰,
通过与LISA和太极的灵敏度曲线比较,发现诱导引力波有望被LISA和太极观测到。

在未来的研究中,将深入研究拐点暴涨模型的构造,原初黑洞的产生,以及诱导引力波的产生。
探讨原初黑洞和诱导引力波的相互检验。

科学的道路无止境,期待宇宙学在不久的将来能够取得突破性的进展,将人类对宇宙的认识
提升到一个新的高度。

