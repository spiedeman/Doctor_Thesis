\section{暴胀宇宙学}
\subsection{大爆炸宇宙学的几大疑难}
大爆炸宇宙学

\section{暴胀模型}
暴胀的基本图像是宇宙在极早期经历过一个加速膨胀的过程,之后转为减速膨胀,过渡到大爆炸宇宙模型。

当暴胀场用标量场$\varphi$描述时,能量密度为
\begin{equation}\label{eq:energy_density}
    \rho = \frac{1}{2}\dot{\varphi}^2+V(\varphi)
\end{equation}
以及压强
\begin{equation}\label{eq:pressure}
    p=\frac{1}{2}\dot{\varphi}^2-V(\varphi)
\end{equation}

相应的连续性方程$(\ref{eq:continuation})$
以及弗里德曼方程$(\ref{eq:1st_friedmann_equation})$改写为
\begin{equation}
    \ddot{\varphi}+3H\dot\varphi+V_{,\varphi}=0 
\end{equation}
\begin{equation}
    H^2=\frac{8\pi G}{3}\left(\frac{1}{2}\dot\varphi^2+V(\varphi)\right)
\end{equation}
约定约化普朗克质量$M_p\equiv\frac{1}{\sqrt{8\pi G}}=1$。消去标量场$\phi$的二阶导数,得到Hamilton-Jacobi方程
    \begin{align}
        \lbrack H'(\varphi)\rbrack^2 - \frac{3}{2}H^2 &=
        -\frac{1}{2}V(\varphi)\label{HJa} \\
        \dot\varphi  &= -2H'(\varphi)\label{HJb}
    \end{align}

    由$(\ref{HJb})$可知$\dot H=-\dot \varphi^2/2\leq 0$,故物理哈勃半径$1/H$随时间增大。同时尺度因子$a(t)$加速增大,故共动哈勃半径$1/(aH)$不断随时间减小。
    方程$(\ref{HJa})$有一个很好的性质,与任意一个方程解的线性扰动都会以指数的速度趋于零,因此很容易进行数值求解。
