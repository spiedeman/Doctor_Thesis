\section{暴胀宇宙学}
大爆炸宇宙学能很好地描述宇宙的演化和大尺度结构的形成,但是也遇到了一些难以解释的疑难。例如平坦性疑难、视界疑难、磁单极子疑难等。

\subsection{视界疑难}
因为取光速$c=1$,所以在物理时间$dt$内,光子走过的固有距离为$cdt=dt$。一阶近似下光子走过的共动距离为$d\eta=dt
/a(t)$,故截止到时刻$t$为止光子走过的最大共动距离为
\begin{align}
  \label{eq:conformal_time}
  \eta = \int_0^{t} \frac{dt^{\prime}}{a(t^{\prime})}.
\end{align}
对应的固有距离为
\begin{align}
  \label{eq:particle_horizon}
  d_{p}(t) = a(t)\eta=a(t)\int_0^{t}\frac{dt^{\prime}}{a(t^{\prime})}.
\end{align}
假设空间中有一点P,到时刻$t$为止能接收到的以光速传播的信号的最远发射源与P相距$d_{\text{max}}=d_{p}(t)$。因此$d_{p}(t)$能够将宇宙分割为以P为中心的可观测和不可观测的两部分。
若有另外一点Q,处于点P的过去光锥(以P为中心的可观测宇宙)外,那么P、Q之间不可能产生过任何相互作用,互相处于对方的因果联系区外。我们将决定因果联系和非因果联系区边界的临界距离$d_{p}(t)$定义为\textbf{粒子视界}。

相对的光在时刻$t$出发所能到达的最远距离被定义为\textbf{事件视界},把宇宙分隔为了$t$时刻信息可影响和不可影响的两部分。用$d_e(t)$表示为
\begin{align}
  d_e(t) = a(t)\chi_e(t) = a(t)\int_t^{\infty}\frac{dt}{a}.
\end{align}
需要注意的是可影响不是立即也非只当未来无穷远时才能产生因果联系,而是取决于相对距离。

在物质/辐射为主时期,粒子视界分别为
\begin{align}
  \label{eq:particle_horizon_matter}
  d_p(t) =
  \begin{cases}
    2H_0^{-1}{\left(\frac{a}{a_0}\right)}^{\frac{3}{2}} =
    2H_0^{-1}{\left(1+z\right)}^{-\frac{3}{2}},\qquad&\text{物质为主},\\
    H_0^{-1}{\left(\frac{a}{a_0}\right)}^2=H_0^{-1}{\left(1+z\right)}^{-2},\qquad&\text{辐射为主}.
  \end{cases}
\end{align}
现在观测到的宇宙尺寸为$d_p(t_0)=2H_0^{-1}$,在红移$z < z_{eq}$(即物质为主时期)时,
目前的可观测宇宙的尺寸要乘以因子${\left(1+z\right)}^{-1}$,
而当时的粒子视界多了因子${\left(1+z\right)}^{-3 /2}$。
因此那时存在${\left(1+z\right)}^{3/2}$个因果逻辑区,互相之间还不能交换信息。并且越早期红移越大,目前的可观测宇宙会
被分割为更多的因果逻辑区。宇宙缘何能在大尺度上实现各向同性的问题就越严重。这就是\textbf{视界疑难}。

\subsection{平坦性疑难}
上一节根据目前的实验观测我们假设代表空间曲率的$K=0$,认为我们所处的宇宙空间是平坦的。但细究之下,在大爆炸宇宙学中,简单的假设$K=0$会带来精细调节的问题。

曲率密度$\Omega_K(z) =
1-\Omega_{tot}=-K /{\left(Ha\right)}^2=\Omega_{K
0}{\left(1+z\right)}^2
/E^2(z)$。因此宇宙永远处于\textbf{平坦、开放、封闭}三种可能性之一,无法互相转换。物质为主时期,$E^2(z)={\left(H
/H_0\right)}^2={\left(1+z\right)}^3$,故$\Omega_K\sim \Omega_{K
0}
/{\left(1+z\right)}$。辐射为主时期,$E^2(z)={\left(1+z\right)}^{4}$,故$\Omega_K\sim\Omega_{K
0}
/{\left(1+z\right)}^2$。不论哪个时期,都表明越接近宇宙早期,曲率密度越接近$1$。例如,在物质-辐射相等时,$\Omega_K\sim
10^{-4}\Omega_{K 0}$;原初核合成时期,$\Omega_K\sim 10^{-16}\Omega_{K
0}$。然而大爆炸理论无法回答曲率密度为何会如此极端的靠近$0$,也无法在理论上给出$K=0$。这就是\textbf{平坦性疑难}。

\subsection{大尺度结构疑难}
在最后散射面上观测到的各向异性,其幅度较小且接近标度不变。在大爆炸到最后散射面形成的这段时间内,标准宇宙学难以解释这些涨落是如何能扩散到这样大的尺度上。
