\section{线性扰动理论}
% 基于广义相对论的宇宙学线性扰动理论在过去的40年中已经发展到高度复杂的程度\citep{bardeen1980gauge,kodama1984cosmological,mukhanov1988quantum}。
% 
% 规范不变扰动理论的一般性框架\citep{sonego1998gauge,nakamura2003gauge,nakamura2005second,bruni1997perturbations,nakamura2006gauge,nakamura2007gauge,nakamura2007second,sachs1964gravitational,stewart1974perturbations,stewart1990perturbations,stewart1993advanced,matarrese1998relativistic,bruni2003two,sopuerta2004nonlinear}

\subsection{扰动分解}
宇宙学原理假设其中之一是宇宙在大尺度结构上是均匀且各向同性。但真实情况是存在一个一阶扰动修正,扰动可以非均匀、各向异性。带有扰动的弗里德曼平坦宇宙的度规为
\begin{align}
	\label{eq:perturbation-metric}
	ds^2=\left[^{(0)}g_{\alpha\beta}+\delta g_{\alpha\beta}(x^{\gamma})
	\right]dx^{\alpha}dx^{\beta},
\end{align}
其中$|\delta g_{\alpha\beta}|\ll|^{(0)}g_{\alpha\beta}|$。采用共动时间时,
背景度规为
\begin{align}
	\label{eq:background-metric}
	^{(0)}g_{\alpha\beta}=a^2(\eta)\text{diag}(1, -1, -1, -1).
\end{align}

我们知道矢量$\bm{V}$可以唯一分解为横向(有旋无源)和纵向(有源无旋)两部分
\begin{align}
  \label{eq:composition-of-vector}
  \bm{V}=\bm{V}_{\parallel}+\bm{V}_{\bot}.
\end{align}

度规扰动$\delta
g_{\alpha\beta}$也存在类似的分解\citep{york1974covariant,deser1967covariant}
,存在\textbf{标量、矢量、张量}三种扰动类型,互相独立。
\begin{align}
\label{eq:decomposition-of-metric}
\delta g_{00} &= 2a^2\phi, \\
\delta g_{0i} &= a^2\lrp{B_{,i}+S_{i}}, \\
\delta g_{ij}&= a^2
\lrp{2\psi \delta_{ij}+2E_{,ij}+F_{i,j}+F_{j,i}+h_{,ij}}.
\end{align}
上式中$\phi$、$B$、$\psi$和$E$为3-标量函数,贡献为标量扰动。$\bm{S}$和$\bm{F}$贡献为矢量扰动,满足无源条件
$S^{i}_{i}=0$以及$F^{i}_{,i}=0$。最后张量扰动来自于横向无迹张量$h_{ij}$,满足四个约束条件
\begin{equation}
  \label{eq:traceless-transerse-tensor-perturbation}
  h^{i}_{i}=0,\quad h^{i}_{j,i}=0.
\end{equation}

通过分析自由度可知,三种扰动类型各自包含两个物理自由度。因而代表标量扰动的四个
函数中必然可以通过规范选择消除其中的两个非物理自由度,矢量扰动和张量扰动类似。

\subsection{扰动的规范不变量}


扰动$\delta g_{\alpha\beta}$的来源之一,是
从物理角度,仅仅因为坐标变换导致的扰动不是真实的扰动。而在坐标变换下的不变量才是反映真实扰动的物理量,
这样的不变量称为\textbf{规范不变量}。以时空为变量的无穷小函数$\xi^{\alpha}(x)$,可以用来定义一个无穷小坐标变换,
\begin{align}
	\label{eq:coordinate-transformation}
	x^{\alpha} \rightarrow \tilde{x}^{\alpha}=x^{\alpha}+\xi^{\alpha}.
\end{align}
对无穷小矢量$\xi^{\alpha}=(\xi^{0}, \xi^{i})$的空间部分进行分解,可以将
其表示为
\begin{equation}
  \label{eq:decomposition-of-xi}
  \xi^{i}=\xi^{i}_{\bot}+\zeta^{,i},
\end{equation}
其中,$\xi^{i}_{\bot}$为横向无源部分,满足条件$\xi^{i}_{\bot,i}=0$,纵向无旋
部分由标量函数$\zeta$表示。

4-标量$q(x^{\rho})$的扰动$\delta
	q=q(x^{\rho})-^{(0)}q(x^{\rho})$按如下方式变换
\begin{align}
	\label{eq:scalar-perturbation-transformation}
	\delta q \rightarrow \delta\tilde{q}=\delta q
	-^{(0)}q_{,\alpha}\xi^{\alpha}.
\end{align}
同理可知4-矢量扰动和4-张量扰动的变换规则为
\begin{align}
	\label{eq:vector-perturbation-transformation}
	\delta u_{\alpha} \rightarrow \delta\tilde{u}_{\alpha}          & =
	\delta
	u_{\alpha}-^{(0)}u_{\alpha,\gamma}\xi^{\gamma}-^{(0)}u_{\gamma}\xi^{\gamma}_{,\alpha},
	\\
	\label{eq:tensor-perturbation-transformation}
	\delta g_{\alpha\beta}\rightarrow\delta \tilde{g}_{\alpha\beta} & =
	\delta g_{\alpha\beta}-^{(0)}g_{\alpha\beta,\gamma}\xi^{\gamma}
	-^{(0)}g_{\gamma\beta}\xi^{\gamma}_{,\alpha}-^{(0)}g_{\alpha\gamma}\xi^{\gamma}_{,\beta}.
\end{align}
根据上一小节,只考虑标量扰动时,度规的最一般形式为
\begin{align}
	\label{eq:scalar-metric-perturbation}
	g_{\alpha\beta}=a^2(\eta)
	\begin{pmatrix}
		1 + 2\phi & B_{,j}                        \\
		B_{,i}    & -(1-2\psi)\delta_{ij}+E_{,ij}
	\end{pmatrix},
\end{align}
在规范变换下,这几个标量函数相应的变换规则为
\begin{align}
  \label{eq:transformation-for-scalar-perturbation-functions}
  \phi &\rightarrow
  \tilde{\phi}=\phi-\frac{1}{a}\lrp{a\xi^{0}}^{\prime}, \\
   B &\rightarrow \tilde{B}=B+\zeta^\prime-\xi^{0}, \\
  \psi&\rightarrow \tilde{\psi}=\psi + \frac{a^\prime}{a}\xi^{0}, \\
   E &\rightarrow \tilde{E}=E+\zeta.
\end{align}
观察发现线性组合这些标量函数可以得到两个最简单的规范不变量,分别为
\begin{align}
	\label{eq:gauge-invariant-scalar}
	\Phi & \equiv \phi
    -\frac{1}{a}{[a{\left(B-E^{\prime}\right)}]}^{\prime}, \\
	\Psi & \equiv \psi + \frac{a^{\prime}}{a}{\left(B-E^{\prime}\right)}.
\end{align}
容易验证$\Phi$和$\Psi$在坐标变换下不变,包含了真实的扰动信息。这也与上一小节
通过分析自由度得到的结论相符。

同理,分析可得矢量扰动和张量扰动的规范不变量分别为$\bar{V_{i}}=S_{i}-F^\prime_{i}$和$h_{ij}$,且物理自由度均为$2$。
因为存在两个非物理自由度,因此可以引入两个约束条件以固定其中两个变量,从而简化方程的求解。通常采用\textbf{纵向规范}和\textbf{同步规范}。

\textbf{同步规范}:$\phi=B=0$。度规的扰动形式由$\phi$、$E$两个标量函数刻画,
\begin{align}
	\label{eq:synchronous-gauge-metric}
	ds^2=a^2{\left[-((1-2\phi)\delta_{ij}-2E_{,ij})dx^{i}dx^{j}\right]}.
\end{align}
由式$(\ref{eq:gauge-invariant-scalar})$可知在同步规范下,规范不变量与扰动之间的关系为
\begin{align}
	\Phi & =\frac{1}{a}{\left[aE^\prime\right]}^{\prime}, \\
	\Psi & =\phi-\frac{a^\prime}{a}E^\prime.
\end{align}

\textbf{纵向规范}:$B=E=0$。度规的扰动形式由$\phi$、$\psi$两个标量函数刻画,
\begin{align}
	ds^2=a^2{\left[(1+2\phi)d\eta^2-(1-2\psi)\delta_{ij}dx^{i}dx^{j}\right]}.
\end{align}
由式$(\ref{eq:gauge-invariant-scalar})$可知在纵向规范下,规范不变量与扰动之间的关系为
\begin{align}
	\Phi & = \phi,  \\
	\Psi & = \psi.
\end{align}
纵向规范下规范不变量与扰动之间的简单关系,使得有关不变量的复杂计算可以在特定的规范下完成。

\subsection{扰动方程}
爱因斯坦场方程是一组非线性方程,在数学上难以求解。当在均匀且各向同性的基础上考虑扰动时,一般也只考虑到方程的线性阶。幸运的是,即便只是精确到线性阶,也足以描述能量密度扰动的非线性演化。

根据场方程,扰动所满足的规范不变场方程为
\begin{align}
	\label{eq:gauge-invariant-perturbation-equation}
	\overline{\delta G}^{\alpha}_{~\beta}=\overline{\delta
		T}^{\alpha}_{~\beta}.
\end{align}
根据之前小节的讨论,$\overline{\delta
		T}^{\alpha}_{~\beta}$
可以分解为标量、矢量、张量三个互不影响的部分。在FLRW度规的背景下,直接计算$\overline{\delta
		G}^{\alpha}_{~\beta}$得到各个扰动分别满足的方程组

\textbf{标量扰动:}
\begin{align}
  \Delta\Phi-3\mathcal{H}{\left(\Phi^\prime+\mathcal{H}\Psi\right)} =
  \frac{1}{2}a^2\overline{\delta T}^{0}_{\ 0}&, 
  \label{eq:equation-gauge-scalar-perturbation1}\\
  {\left(\Phi^\prime+\mathcal{H}\Phi\right)}_{,i}=\frac{1}{2}
  a^2\overline{\delta T}^{0}_{\ i}&,
  \label{eq:equation-gauge-scalar-perturbation2}\\
    {\left[\Phi^{\prime\prime}+\mathcal{H}{\left(2\Psi+\Phi\right)}^{\prime}+(2\mathcal{H^\prime}+\mathcal{H}^2)\Phi 
+\frac{1}{2}\Delta(\Phi-\Psi)\right]}&\delta_{ij}\notag \\
    -\frac{1}{2}{\left((\Phi-\Psi)\right)}_{,ij} = -\frac{1}{2}
a^2\overline{\delta T}^{i}_{\ j}&,
  \label{eq:equation-gauge-scalar-perturbation3}
\end{align}
其中,$\mathcal{H}=a^\prime /a$,撇号表示对共形时间$\eta$的导数。

\textbf{矢量扰动}:
\begin{align}
  \label{eq:equation-gauge-vector-perturbation}
  \Delta\overline{V}_{i}=2a^2\overline{\delta T}^{0}_{\ i(V)}, \\  
  {\left(\overline{V}_{i,j}+\overline{V}_{j,i}\right)}^{\prime} + 
  2\mathcal{H}{\left(\overline{V}_{i,j}+\overline{V}_{j,i}\right)} = 
  -2a^2&\overline{\delta T}^{i}_{\ j(V)}.
\end{align}

\textbf{张量扰动}:
\begin{align}
  \label{eq:equation-gauge-tensor-perturbation}
  h^{\prime\prime}_{ij}+2\mathcal{H}h^\prime_{ij}-\Delta h_{ij}=
  2a^2\overline{\delta T}^{i}_{\ j(T)}.
\end{align}
这些方程描述了各种类型的扰动如何随时间演化的过程。物质在宇宙中的分布可以用理想流体来近似描述,而在这种情况下得到的矢量扰动的解正比于尺度因子的负二次方,因此矢量类型的度规扰动会衰减的非常迅速以至于在通常情况下都可以将其忽略。
张量扰动正是所谓的引力波,此处暂且把焦点放在标量扰动上,后文关于第二个工作的内容中会更加详细地讨论二阶张量扰动。

根据理想流体的能动量张量$(\ref{eq:perfect-fluid-energy-momentum-tensor})$,容易得到相应的规范不变量为
\begin{align}
  \label{eq:perfect-fluid-gauge-invariant-variable}
  \overline{\delta{T}}^{0}_{\ 0} &= \overline{\delta{\rho}} \\
  \overline{\delta{T}}^{0}_{\ i} &=
  \frac{1}{a}(\rho_0+p_0){\left(\overline{\delta{u}}_{\shortparallel
        i}+\overline{\delta{u}}_{\perp i}\right)} \\
  \overline{\delta{T}}^{i}_{\ j}&= -\overline{\delta{p}}\delta^{i}_{\ j},
\end{align}
其中,$\overline{\delta{\rho}}$、$\overline{\delta{u}}_{\shortparallel
i}$、$\overline{\delta{p}}$分别是能量密度扰动、无旋速度扰动、压强扰动对应的规范不变量。
标量扰动的贡献只来自于这几项,正比于速度扰动的无源分量$\overline{\delta{u}}_{\perp
i}$的项只对矢量扰动产生贡献。\\
当$i\neq
j$时,$\delta{T}^{i}_{j}=0$,方程$(\ref{eq:equation-gauge-scalar-perturbation3})$约化为
\begin{align}
  {\left(\Phi-\Psi\right)}_{,ij}=0\qquad (i\neq j).
\end{align}
有唯一解$\Phi=\Psi$。从而标量扰动满足的方程组进一步被化简为
\begin{align}
  \Delta \Phi -3\mathcal{H}{\left(\Phi^\prime+\mathcal{H}\Phi\right)} =
  \frac{1}{2}a^2\overline{\delta{\rho}},
  \label{eq:simplified-equation-gauge-scalar-perturbation-1} \\
  {\left(a\Phi\right)}^{\prime}_{,i}=\frac{1}{2}
  a^2{\left(\rho_0+p_0\right)}\overline{\delta{u}}_{\shortparallel i},
  \label{eq:simplified-equation-gauge-scalar-perturbation-2} \\
  \Phi^{\prime\prime}+3\mathcal{H}\Phi^\prime+(2\mathcal{H}^{\prime}+\mathcal{H}^2)\Phi=
  \frac{1}{2}a^2\overline{\delta{p}}.
  \label{eq:simplified-equation-gauge-scalar-perturbation-3}
\end{align}
在非膨胀宇宙中$\mathcal{H}=0$,方程$(\ref{eq:simplified-equation-gauge-scalar-perturbation-1})$与引力势满足的泊松方程
相同。在膨胀宇宙中,方程$(\ref{eq:simplified-equation-gauge-scalar-perturbation-1})$左边的第二、三项在视界内会被压低一个因子
$\sim\lambda
/H^{-1}$,可以忽略。故而方程$(\ref{eq:simplified-equation-gauge-scalar-perturbation-1})$可以看作是泊松方程的推广,而度规的一阶标量扰动$\Phi$为牛顿引力势在广义相对论下的推广。

从热力学可知,压强作为内能和熵的函数$p(\rho, S)$,其涨落的表达式为
\begin{align}
  \label{eq:perturbation-of-pressure}
  \overline{\delta
  p}=c^2_{s}\overline{\delta\rho}+\tau\delta S,
\end{align}
其中$c^2_{s}\equiv {\left(\partial p /\partial
\rho\right)}_{S}$为声速,$\tau\equiv{\left(\partial p /\partial
S\right)}_{\rho}$。将上式带入方程$(\ref{eq:simplified-equation-gauge-scalar-perturbation-1})$和$(\ref{eq:simplified-equation-gauge-scalar-perturbation-3})$可得关于引力势的闭方程
\begin{align}
  \label{eq:closed-form-equation-for-Phi}
  \Phi^{\prime\prime}+3{\left(1+c^2_{s}\right)}\mathcal{H}\Phi^\prime-c^2_{s}\Delta\Phi+{\left(2\mathcal{H}^{\prime}+
  {\left(1+3c^2_{s}\right)}\mathcal{H}^2\right)}\Phi=\frac{1}{2}a^2\tau\delta S.
\end{align}
针对上述方程,共有8种情况需要讨论。是否为绝热扰动${\left(\delta
S=0\right)}$,是否为极端相对论物质,以及长短波极限。即便在绝热扰动下,
上述方程也无法对任意物态方程$p(\rho)$给出解析解,不过针对长波及短波极限能够给出相应的渐进解。这里仅以极端
相对论物质的绝热扰动为例略加讨论。
\\
物态方程为
\begin{align}
  p=w\rho.
\end{align}
因此声速为
\begin{align}
  c^2_{s}=w.
\end{align}
宇宙标度因子为
\begin{align}
  a\propto\eta^{2 /(1+3w)}.
\end{align}
结合上面两式,方程$(\ref{eq:closed-form-equation-for-Phi})$在傅里叶空间中变为
\begin{align}
  \label{eq:adiabatic-relativistic-equation-of-Phi}
  \Phi^{\prime\prime}_{\mathbf{k}}+\frac{6(1+w)}{1+3w}\frac{1}{\eta}\Phi^\prime_{\mathbf{k}}+wk^2\Phi_{\mathbf{k}}=0.
\end{align}
方程有解析解
\begin{align}
  \label{eq:solution-for-adiabatic-relativistic-equation-of-Phi}
  \Phi_{\mathbf{k}}=\eta^{-\nu}{\left[
  C_1J_{\nu}{\left(\sqrt{w}k\eta\right)}+C_2Y_{\nu}{\left(\sqrt{w}k\eta\right)}
  \right]},\quad
  \nu\equiv \frac{1}{2}{\left(\frac{5+3w}{1+3w}\right)},
\end{align}
其中$J_{\nu}$和$Y_{\nu}$为$\nu$阶贝塞尔函数,分别为扰动$\Phi$的非衰减模式和衰减模式。\\
联合$(\ref{eq:simplified-equation-gauge-scalar-perturbation-1})$和方程
\begin{align}
  ^{(0)}G^{0}_{~0} =^{(0)}T^{0}_{~0}, 
\end{align}
得到能量密度的相对扰动
\begin{align}
  \label{eq:gauge-invariant-relative-energy-density-perturbation}
  \delta\equiv\frac{\overline{\delta\rho}}{\rho_0}
  =\frac{k^2 \Phi -3\mathcal{H}{\left(\Phi^\prime+\mathcal{H}\Phi\right)}}{\frac{3}{2}\mathcal{H}^2}.
\end{align}
在\textbf{长波极限}下,$\sqrt{w}k\eta\ll
1$,$\delta$及$\Phi$的非衰减模式均趋近于常数,且
\begin{align}
  \overline{\delta\rho} /\rho_0\simeq -2\Phi.
\end{align}
说明不论是度规扰动还是能量密度的相对扰动,在离开视界之后都会趋近于稳定。\\
在\textbf{短波极限}下,$\sqrt{w}k\eta\gg 1$,即对尺度小于Jeans波长($\lambda_{J}\sim
c_{s}t$)的扰动模式而言,其行为更像是振幅随时间衰减的声波
\begin{align}
  \Phi_{\mathbf{k}}\propto \eta^{-\nu-\frac{1}{2}}\exp{\left(\pm
  i\sqrt{w}k\eta\right)}. 
\end{align}
特别是在\textbf{辐射为主}时期,$w=1 /3$以及$\nu=1
/3$,贝塞尔函数有更简单的表达式使得
\begin{align}
  \label{eq:Phi-in-radiation-dominated}
  \Phi_{\mathbf{k}}=\frac{1}{x^2}{\left[
  C_1{\left(\frac{\sin x}{x}-\cos x\right)}+
  C_2{\left(\frac{\cos x}{x}+\sin x\right)}\right]}.
\end{align}
其中$x\equiv k\eta /\sqrt{3}$。对应的相对能量密度扰动为
\begin{align}
  \frac{\overline{\delta\rho}}{\rho_0}=
  &2C_1{\left[{\left(\frac{2-x^2}{x^2}\right)}{\left(\frac{\sin x}{x}-\cos
  x\right)}-\frac{\sin x}{x}\right]} \notag
  \\
  &+4C_2{\left[{\left(\frac{1-x^2}{x^2}{\left(\frac{\cos x}{x}+\sin x\right)}+\frac{\sin x}{2}\right)}\right]}.
\end{align}
通常会像上面方式一样在特定条件下简化方程,再求解。虽然无法在任意情况下给出解析解,
不过仍然可以通过引入变量的方式,将
绝热扰动情况下($\delta S=0$)引力势满足的方程$(\ref{eq:closed-form-equation-for-Phi})$化简为如下形式
\begin{equation}
  \label{eq:ms-like-bardeen-equation}
  u^{\dprime}-c_{s}^2\Delta u - \frac{\theta^{\dprime}}{\theta} u = 0. 
\end{equation}
式中的两个变量$u$和$\theta$均为规范不变量,定义如下
\begin{align}
  \label{eq:u-theta-defininition}
  u &\equiv \exp\lrp{\frac{3}{2}\int (1+c_{s}^2)\mathcal{H}d\eta}\Phi 
      =\frac{\Phi}{\lrp{\rho_0+p_0}^{1 /2}}, \\
  \theta &\equiv \frac{1}{a}\lrp{1+\frac{p_0}{\rho_0}}^{-1 /2}=
  \frac{1}{a}\lrp{\frac{2}{3}\lrp{1-\frac{\mathcal{H^\prime}}{\mathcal{H}^2}}}^{-1
  /2}.
\end{align}

求解$\Phi$现在变成对$u$的求解,散度项的存在妨碍了对方程$(\ref{eq:ms-like-bardeen-equation})$的求解。
变换到傅立叶空间后,发现散度项的大小反映了k模的波长。因而若取长波近似,则
散度项可以被安全的略去。
\begin{equation}
  u^{\dprime} - \frac{\theta^{\dprime}}{\theta} u =0. 
\end{equation}
显然此时$u\propto \theta$是方程的一个解,利用Wronskian方法可以得到第二个解
\begin{equation}
  u\simeq C_1\theta +C_2\theta \int_{\eta_0}\frac{d\eta}{\theta^2}
  =C_2\theta \int_{\tilde{\eta_0}}\frac{d\eta}{\theta^2}.
\end{equation}
通过分部积分
\begin{equation}
  \int \frac{d\eta}{\theta^2}=
  \frac{2}{3}\int a^2\lrb{1+\lrp{\frac{1}{\mathcal{H}}}^{\prime}}d\eta
  =\frac{2}{3}\lrp{\frac{a^2}{\mathcal{H}}-\int a^2d\eta}.
\end{equation}
求得引力势
\begin{equation}
  \Phi=\lrp{\rho_0+p_0}^{1 /2}u
  = A\lrp{1-\frac{\mathcal{H}}{a^2}\int a^2d\eta}
  = A \frac{d}{dt}\lrp{\frac{1}{a}\int adt}.
\end{equation}
式中$A$为积分常数。

\subsection{原初功率谱}
在傅立叶空间中,扰动的某个模式一旦在暴涨过程中离开了视界,则该模式对应的扰动将不再变化,直到再次进入视界之后才会开始二度演化。
因此暴涨结束后扰动在傅立叶空间中的分布,即功率谱将成为扰动再次进入视界时演化的初值。

以张量扰动为例,$h_{ij}$横向无迹,自由度为2,由函数$h_{+}$和$h_{\times}$分别描述$+$模式和$\times$模式。两个模式服从相同的扰动方程
\begin{equation}
  \label{eq:tensor-perturbation-equation}
  h^{\prime\prime}+ 2\mathcal{H}h^\prime+k^2h = 0,
\end{equation}
通过变量代换$\tilde{h}=ah/\sqrt{2}$,化简为关于$\tilde{h}$的简谐方程
\begin{equation}
  \tilde{h}^{\prime\prime}+(k^2-\frac{a^{\prime\prime}}{a})\tilde{h} =0. 
\end{equation}
对$\tilde{h}$进行量子化,
\begin{equation}
  \hat{\tilde{h}}(\vec{k},\eta)=v_k(\eta)\hat{a}_{\vec{k}}+v^{*}_k(
  \eta)a^{\dagger}_{\vec{k}},
\end{equation}
其中产生、湮灭算符的系数满足相同的简谐方程
\begin{equation}
  \label{eq:harmonic-equation}
  v_k^{\prime\prime}+(k^2-\frac{a^{\prime\prime}}{a})v_k =0. 
\end{equation}
场算符$\tilde{h}$方差的期望值为
\begin{equation}
  \langle \hat{\tilde{h}}^{\dagger}(\vec{k},
  \eta)\hat{\tilde{h}}(\vec{k}^{\prime},\eta)\rangle={\lvert
  v_k(\eta)\rvert}^2 {\left(2\pi
\right)}^3\delta^{3}(\vec{k}-\vec{k^\prime}).
\end{equation}
因此扰动$h$量子化后对应的场的方差的期望为
\begin{align}
  \langle \hat{h}^{\dagger}(\vec{k},\eta)\hat{h}(\vec{k}^{\prime},\eta) \rangle
  &= \frac{2}{a^2}{\lvert
  v_k(\eta)\rvert}^2{\left(2\pi\right)}^3\delta^3(\vec{k}-\vec{k}^{\prime})
  \notag
  \\
  &\equiv {\left(2\pi\right)}^3 \mathcal{P}_h(k)\delta^3(\vec{k}-\vec{k}^\prime).
\end{align}
第二行定义了原初张量扰动的功率谱
\begin{equation}
  \label{eq:tensor-primodial-pertubation} 
  \mathcal{P}_h(k) = \frac{2}{a^2}{\lvert v_k(\eta)\rvert}^2.
\end{equation}
对任意物态方程参数$w$,由连续性方程$(\ref{eq:continuation})$可知
\begin{equation}
  a \propto \eta^{\frac{2}{1+3w}} = \eta^{r},
\end{equation}
其中,$r=1,2$分别表示辐射为主时期和物质为主时期,暴涨期间近似为德$\cdot$西特宇宙,物态方程参数$w\approx
-1$,故$r\approx -1$,此时方程$(\ref{eq:harmonic-equation})$存在解析解
\begin{equation}
  \label{eq:solution-for-harmonic-equation}
  v_k = \frac{e^{-ik\eta}}{\sqrt{2k}}{\left[1-\frac{i}{k\eta}\right]}.
\end{equation}
因为要计算模式离开视界后的扰动,取长波极限($k\lvert \eta\rvert \ll
1$)代回功率谱的定义中得到
\begin{align}
  \label{eq:tensor-power-spectrum}
  \mathcal{P}_h(k)&= \frac{2}{a^2}\frac{1}{2k^3\eta^2} \notag\\
        &= \frac{H^2}{k^3}.
\end{align}
其中哈勃参数取值为模式$k$离开视界时的值。

标量扰动计算以暴涨场$\varphi(\vec{x},\eta)$为例,将其分解为均匀且各项同性的背景部分以及线性扰动部分
\begin{equation}
  \varphi(\vec{x},\eta)=\varphi_{0}(\eta)+\delta\varphi(\vec{x},\eta), 
\end{equation}
% 为了尽可能简化$\delta\varphi(\vec{x},\eta)$满足的扰动方程,选取spatially flat
% slicing 规范,该规范下度规为
% \begin{equation}
%   \label{eq:spatially-flat-slicing-gauge-metric}
%   g_{\mu\nu} = \begin{pmatrix}
%     1+2\phi & B_{,i} \\
%     B_{,j} & -\delta_{ij}
%   \end{pmatrix}
% \end{equation}
% 扰动$\delta\varphi$在该规范下满足的方程为
% \begin{equation}
%   \delta\varphi^{\prime\prime} + 2
%   \mathcal{H}\delta\varphi^\prime+k^2\delta\varphi = 0. 
% \end{equation}
% 与张量扰动满足的方程$(\ref{eq:tensor-perturbation-equation})$形式相同,因此可直接得到对应的功率谱为
% \begin{equation}
%   \label{eq:power-spectrum-for-delta-varphi}
%   P_{\delta\varphi}=\frac{H^2}{2k^{3}}. 
% \end{equation}
% 扰动$\Phi$的功率谱计算需借助另一个规范不变量——标量曲率扰动$\zeta$。
暴涨过程中,
虽然能动张量不同,不过在引入相同的规范变量$u$和$\theta$后,标量扰动化简后的形式
与$(\ref{eq:ms-like-bardeen-equation})$相同,并且可以进一步改写成形式更为紧凑的下式
\begin{equation}
  \lrb{\theta^2\lrp{\frac{u}{\theta}}^{\prime}}^{\prime} =
  c_{s}^2\theta\Delta u.
\end{equation}
从而在超视界区域我们发现了一个“守恒量”——曲率扰动,
\begin{equation}
  \label{eq:curvature-perturbation-defininition}
  \mathcal{R}\equiv \frac{2}{3}\lrp{\frac{u}{\theta}}^{\prime}\theta^2 = 0.
\end{equation}
将$u$和$\theta$的定义$(\ref{eq:u-theta-defininition})$带入$\mathcal{R}$的定义式
中$(\ref{eq:curvature-perturbation-defininition})$,得到$\mathcal{R}$的表达式为
\begin{equation}
  \mathcal{R} = \Phi +
  \frac{\mathcal{H}}{\varphi^{\prime}_0}\overline{\delta\varphi}.
\end{equation}
在长波极限下,$\Phi$和$\overline{\delta\varphi}$的渐进解分别为
\begin{align}
  \Phi &\simeq A\frac{d}{dt}\lrp{\frac{1}{a}\int a dt  } =
  A\lrp{1-\frac{H}{a}\int adt}, \\
  \overline{\delta\varphi} &\simeq A \dot{\varphi_0}\lrp{\frac{1}{a}\int
  adt}.
\end{align}
通过线性组合上两式,发现积分常数$A$恰好等于$\mathcal{R}$。暴涨结束后,标量场通常进入
振荡阶段,此时标度因子$a$以物理时间$t$的幂指数方式增长,$a\propto
t^{p}$。故
\begin{equation}
  \Phi \simeq \frac{A}{p+1},\quad 
  \overline{\delta\varphi}\simeq \frac{At\dot{\varphi_0}}{p+1},
\end{equation}
标量场最终将转化为极端相对论物质,故幂指数$p=1 /2$,得到
\begin{equation}
  \Phi \simeq \frac{2}{3}A. 
\end{equation}
因此$\Phi$和$\mathcal{R}$的功率谱之间的关系为
\begin{equation}
  \mathcal{P}_{\Phi}(k) = \frac{4}{9}\mathcal{P}_{\mathcal{R}}(k). 
\end{equation}
于是问题转化为了如何求解功率谱$\mathcal{P}_{\mathcal{R}}(k)$。
通过引入两个中间变量$v$和$z$
\begin{align}
  v&\equiv\theta\lrp{\frac{u}{\theta}}^{\prime} 
  =a\lrp{\overline{\delta\varphi} +
  \frac{\varphi^\prime_0}{\mathcal{H}}\Phi}, \\
  z&\equiv \frac{1}{\theta} = \frac{a^2\lrp{\rho_0+p_0}^{1
  /2}}{\mathcal{H}}.
\end{align}
可以将方程$(\ref{eq:ms-like-bardeen-equation})$拆分两个方程
\begin{align}
  c_{s}^2\Delta u=z\lrp{\frac{v}{z}}^{\prime},\\
  v=\theta\lrp{\frac{u}{\theta}}^{\prime}.
\end{align}
并且能使$\mathcal{R}$表示成更简单的形式$\mathcal{R}=v
/z$。利用上面两个方程能够得到$v$满足的方程
\begin{equation}
  \label{eq:ms-equation}
  v^{\dprime}- c_s^2\Delta v - \frac{z^{\dprime}}{z}v = 0.
\end{equation}
恰巧与$u$满足的方程形式相同,为所谓的Mukhanov-Sasaki(MS)方程。完整的求解$v$的功率谱需要在量子场论中构造相应的
作用量,并选取合适的真空态。仅从数学的角度看,可以关注于方程$(\ref{eq:ms-equation})$本身。暴涨期间,采用慢滚近似,则
有$z^{\dprime} /z = (\nu^2-1 /4) /\eta^2$,其中$\nu=3
/2+2\epsilon_H-\eta_H$。故从MS方程得到了一个贝塞尔方程
\begin{equation}
  v^{\dprime}_k +\lrp{k^2-\frac{\nu^2-1 /4}{\eta^2}}v_{k} =0, 
\end{equation}
结合边值条件,得到$v_{k}$的解为
\begin{equation}
  v_{k}(\eta) = \frac{\sqrt{\pi}}{2}e^{i(\nu+1 /2)\pi
  /2}\sqrt{-\eta}H^{(1)}_{\nu}(-k\eta).
\end{equation}
对于超视界的扰动,
\begin{align}
  H^{(1)}_{\nu}(x\ll 1) \sim \sqrt{\frac{2}{\pi}}e^{-i\pi /2}2^{\nu-3 /2}
  \frac{\Gamma(\nu)}{\Gamma(3 /2)}x^{-\nu}, \\
  v_{k}(\eta)=e^{i(\nu-1 /2)\pi /2}2^{\nu-3 /2}
  \frac{\Gamma(\nu)}{\Gamma(3 /2)} \frac{1}{\sqrt{2k}}\lrp{-k\eta}^{1
  /2-\nu}.
\end{align}
最终得到标量扰动的功率谱为
\begin{align}
  \mathcal{P}_{\Phi}(k)=\frac{4}{9}\mathcal{P}_{\mathcal{R}}(k) 
  &= \frac{4}{9}\frac{k^3}{2\pi^2}\lrv{\frac{v_{k}}{z}}^2 \notag\\
               &= \frac{4}{9}2^{2\nu -3}\lrp{\frac{\Gamma(\nu)}{\Gamma(3 /2)}}^2
               \lrp{\frac{H}{\dot{\varphi_0}}}^2
               \lrp{\frac{H}{2\pi}}^2
               \lrp{\frac{k}{aH}}^{3-2\nu}
             \Bigg\rvert_{k=aH}.
\end{align}


% 为了最终得到规范不变的标量扰动$\Phi$的功率谱,利用Bardeen's velocity
% \begin{equation}
%   \label{eq:bardeen's-velocity}
%   v=ikB-\frac{ik \varphi^{(0)\prime}\delta\varphi}{(\rho+p)a^2}\qquad
%   (\text{spatially flat slicing}).
% \end{equation}
% 又因为在 spatially flat slicing
% 规范中,规范不变量$\Psi=aHB$,所以根据式$(\ref{eq:bardeen's-velocity})$得到的组合项
% \begin{equation}
%   \label{eq:gauge-invariant-mathcal{R}}
%   \mathcal{R} \equiv -\Psi- \frac{iaH}{k}v =
%   -\frac{aH}{\varphi^{(0)\prime}}\delta\varphi,
% \end{equation}
% 也是规范不变量,于是$\mathcal{R}$的功率谱与扰动$\delta\varphi$的功率谱之间存在简单的关系
% \begin{equation}
%   P_{\mathcal{R}} =
%   {\left(\frac{aH}{\varphi^{(0)\prime}}\right)}^2P_{\delta\varphi}. 
% \end{equation}
% 经过简单的计算可得${\left(aH /
% \varphi^{(0)\prime}\right)}^2=4\pi
% G/\epsilon$,以及式$(\ref{eq:power-spectrum-for-delta-varphi})$,所以$\mathcal{R}$的功率谱为
% \begin{equation}
%   P_{\mathcal{R}} = \frac{2\pi GH^2}{\epsilon k^3} \mid_{aH=k}.
% \end{equation}
% 由于暴涨结束后,在共性牛顿规范下,有$\mathcal{R}=3\Psi/2$,故而
% \begin{equation}
%   P_{\Psi} = \frac{8\pi GHs^2}{9\epsilon k^3}|_{aH=k}.
% \end{equation}
% 
