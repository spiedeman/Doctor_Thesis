\section{CMB和标量扰动}

宇宙学原理假设其中之一是宇宙在大尺度结构上是均匀且各向同性。但真实情况是存在一个一阶扰动修正,扰动可以非均匀、各向异性。带有扰动的弗里德曼平坦宇宙的度规为
\begin{align}
  \label{eq:perturbation-metric}
  ds^2=\left[^{(0)}g_{\alpha\beta}+\delta g_{\alpha\beta}(x^{\gamma})
  \right]dx^{\alpha}dx^{\beta},
\end{align}
其中$|\delta g_{\alpha\beta}|\ll|^{(0)}g_{\alpha\beta}|$。采用共动时间时,
背景度规为
\begin{align}
  \label{eq:background-metric}
  ^{(0)}g_{\alpha\beta}=a^2(\eta)\text{diag}(1, -1, -1, -1).
\end{align}
度规扰动$\delta
g_{\alpha\beta}$可以分解为三种类型:\textbf{标量扰动}、\textbf{矢量扰动}和\textbf{张量扰动}。
三种扰动独立演化,互不影响。
扰动$\delta g_{\alpha\beta}$的来源之一,是
从物理角度,仅仅因为坐标变换导致的扰动不是真实的扰动。而在坐标变换下的不变量才是反映真实扰动的物理量,
这样的不变量称为\textbf{规范不变量}。以时空为变量的无穷小函数$\xi^{\alpha}(x)$,可以用来定义一个无穷小坐标变换。
\begin{align}
  \label{eq:coordinate-transformation}
  x^{\alpha} \rightarrow \tilde{x}^{\alpha}=x^{\alpha}+\xi^{\alpha}.
\end{align}
4-标量$q(x^{\rho})$的扰动$\delta
q=q(x^{\rho})-^{(0)}q(x^{\rho})$按如下方式变换
\begin{align}
  \label{eq:scalar-perturbation-transformation}
  \delta q \rightarrow \delta\tilde{q}=\delta q
  -^{(0)}q_{,\alpha}\xi^{\alpha}.
\end{align}
同理可知4-矢量扰动和4-张量扰动的变换规则为
\begin{align}
  \label{eq:vector-perturbation-transformation}
  \delta u_{\alpha} \rightarrow \delta\tilde{u}_{\alpha}&=
  \delta
  u_{\alpha}-^{(0)}u_{\alpha,\gamma}\xi^{\gamma}-^{(0)}u_{\gamma}\xi^{\gamma}_{,\alpha},
  \\
  \label{eq:tensor-perturbation-transformation}
  \delta g_{\alpha\beta}\rightarrow\delta \tilde{g}_{\alpha\beta} &=
  \delta g_{\alpha\beta}-^{(0)}g_{\alpha\beta,\gamma}\xi^{\gamma}
  -^{(0)}g_{\gamma\beta}\xi^{\gamma}_{,\alpha}-^{(0)}g_{\alpha\gamma}\xi^{\gamma}_{,\beta}.
\end{align}

考虑标量扰动时,度规的最一般形式为
\begin{align}
  \label{eq:scalar-metric-perturbation}
  g_{\alpha\beta}=a^2(\eta)
  \begin{pmatrix}
    1 + 2\phi & B_{,j} \\
    B_{,i} & -(1-2\psi)\delta_{ij}+E_{,ij}
  \end{pmatrix}
\end{align}

其中$\phi$,$\psi$,$B$,$E$为3-标量。
根据扰动的规范变换,可得这几个标量函数相应的变换规则。通过观察发现有这些标量函数的线性组合可得的最简单的
规范不变量有两个,分别为
\begin{align}
  \label{eq:gauge-invariant-scalar}
  \Phi\equiv \phi -\frac{1}{a}{[a{\left(B-E^{\prime}\right)}]}^{\prime}
  ,\qquad
  \Psi \equiv \psi + \frac{a^{\prime}}{a}{\left(B-E^{\prime}\right)}.
\end{align}
容易验证$\Phi$和$\Psi$在坐标变换下不变,包含了真实的扰动信息。

同理可得,矢量扰动和张量扰动的规范不变量为$\bar{V_{i}}=S_{i}-F^\prime_{i}$和$h_{ij}$,且物理自由度均为$2$。
因为存在两个非物理自由度,因此可以引入两个约束条件以固定其中两个变量,从而简化方程的求解。通常
