\section{超引力中的暴胀模型简介}
在超引力中嵌入暴胀模型的研究已经超过三十年了,第一个在超引力中实现的暴胀模型——
混沌暴胀可追溯之1983年
\citep{goncharov1984chaotic},与混沌暴胀的一般化模型的提出仅相隔几个月。 
这一章论述在超引力中利用平移对称性构造跑动动能暴胀模型,并以此为出发点对场进行
正则归一化,从而得到符合拐点暴胀模型的势能函数。

基于超引力构建的大多数旧暴胀模型\citep{freedman1989progress,deser1976consistent,wess1992supersymmetry}都遇到了$\eta$问题\citep{yamaguchi2011supergravity}:
由于F项势中包含$e$指数因子$e^{{\lvert
\Phi\rvert}^2}$,这一因子使得标量场获得一个与哈勃参数同量级的质量项,这就使得慢滚参数之一的$\eta$变的大于一,从而破坏了慢滚条件,标量场将迅速
滚到势能的极小值处而不能形成合理的暴胀过程。
如果是小场暴胀模型的话,还可以通过调节参数使得势能在某个区间变得较为平坦,从而
使暴胀过程顺利发生。但对于大场暴胀模型而言,调节参数的方法几乎不可行。最简单的
解决$\eta$问题的方法是将平移对称性(shift symmetry)引入K\"ahler势中\citep{kawasaki2000natural,kawasaki2001natural}。
不过直接引入平移对称性,会导致势能不存在下确界的问题\citep{kawasaki2000natural,kawasaki2001natural}。

除了通过调节参数的方法以外\citep{linde2015single,roest2015cosmological,goncharov1984chaotic},有两种一般性的方法可以解决势能无下确界的问题。
第一种方法是引入额外的稳定场$S$\citep{kawasaki2000natural,kawasaki2001natural,kallosh2010new,kallosh2011general}。第二种方法是将满足平移对称性的暴胀场的4次方项
引入K\"ahler势中\citep{ketov2016single,ketov2014inflation,izawa2007supersymmetric,ketov2014generic}。

早期有一类双场理论
\citep{kallosh2010new},$S=s
e^{i\theta}/\sqrt{2}$和$\Phi=(\phi+i\chi)/\sqrt{2}$,以及超势$W=Sf(\Phi)$,其中$f(\Phi)$是实全纯函数,满足$\bar{f}(\Phi)=f(\Phi)$。
模型中有两个超场,或者四个实标量场,$s$,$\theta$,$\phi$和$\chi$,其中只有
一个实场对应暴胀场,参与宇宙学演化。通常选取$\phi$或$\chi$作为暴胀场,$S$作为辅助场。
K\"ahler势的函数形式可以取做$K=K({(\Phi-\bar{\Phi})}^2,S\bar{S})$。这种情形下,势能$V$在$s=\chi=0$处有极值,如果该极值同时也
是全局最小值,那么标量场$\phi$将扮演正则归一化的暴胀场,势能为$V(\phi)=|f(\phi/\sqrt{2})|^2$,且暴胀过程发生在$s=\chi=0$的方向上。为了与宇宙学观测保持一致,
暴胀模型需要给出符合观测的宇宙学参数$n_{s}$和$r$,这一点总是可以通过选取合适
的函数$f(\Phi)$来做到。
