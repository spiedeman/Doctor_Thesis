\section{原初黑洞}
当某个原初密度扰动足够大的尺度在暴涨结束后重新进入视界时,由引力塌缩可能形成原初黑洞。
一般黑洞的质量$M$正比于当时的一个哈勃体积内的总质量$M_H$,比例系数为$\gamma$。
\begin{equation}
    M = \gamma M_H = \gamma \frac{4}{3}\pi\rho H^{-3}
\end{equation}
系数$\gamma$取决于引力塌缩的过程,与采取的模型有关,一般取
$0.2$在辐射为主时期\citep{carr1975primordial}。利用熵守恒$d(g_s(T)T^3a^3)/dt=0$和$\rho\propto
g(T)T^4$,可以得到辐射为主时期:
\begin{equation}
    \label{eq:mass_pbh}
    M=\gamma M_{H(eq)}{\left(\frac{g(T_f)}{g(T_{eq})}\right)}^{1/2}
    {\left(\frac{g_s(T_f)}{g_s(T_{eq})}\right)}^{-2/3}
    {\left(\frac{k}{k_{eq}}\right)}^{-2}
\end{equation}

$M_{H(eq)}$表示辐射-物质相等时的视界内的总质量。因为物质为主时期
$H^2\propto \rho \propto a^{-3}$和 $k=aH$,故有
\begin{align*}
    \label{eq:horizon_mass_eq}
    M_{H(eq)} &= \frac{4}{3}\pi\rho_{eq}H_{eq}^{-3} \\
    &= \frac{4}{3}\pi\rho_m
    a_{eq}^{-3}{\left(\frac{k_{eq}}{a_{eq}}\right)}^{-3}\\
    &= \frac{4}{3}\pi \Omega_m \rho_{0,crit}k^{-3}_{eq} \\
    &\approx 3\times 10^{50}\ g
\end{align*}

在辐射为主时期假设 $g(T)=g_s(T)$
是一个好的近似,以及最新Planck
2018数据\citep{aghanim2018planck}给出的$k_{eq}=0.073\Omega_m
h^2\ \text{Mpc}^{-1}$。于是{(\ref{eq:mass_pbh})}可写成
\begin{equation}
    M(k) =
    10^{18}g\left(\frac{\gamma}{0.2}\right){\left(\frac{g(T_f)}{106.75}\right)}^{-1/6}
    {\left(\frac{k}{7\times10^{13}\ \text{Mpc}^{-1}}\right)}^{-2}
\end{equation}
$M(k)$表示共动波数$k$重新进入视界时形成的黑洞的质量。

在Press-schechter引力塌缩模型中\citep{press1974formation},质量M的原初黑洞的生成率由一个高斯随机概率分布决定。当相对密度扰动$\delta$大于某个阈值$\delta_c$时,视界内的物质将在引力的作用下塌缩形成一个黑洞,那么质量为M的原初黑洞所占丰度$\beta(M)$将由高斯分布的分布函数给出:
\begin{align}
    \label{eq:mass_fraction_pbh}
    \beta(M) = \frac{1}{\sqrt{2\pi \sigma^2(M)}}\int_{\delta_c}^{\infty}
    d\delta\ \text{exp}\left(\frac{-\delta^2}{2\sigma^2(M)}\right).
\end{align}

唯一不确定的还剩下方差$\sigma^2(M)$。对它的计算一般采用如下方式,将空间粗粒化成一个个尺度为$R=1/k$的区域,并用高斯窗口函数$W(x)=\text{exp}(-x^2/2)$将相对密度扰动光滑化,最后对所有区域求$\delta$的方差作为$\sigma^2(M)$:
\begin{align}
    \sigma^2(M(k)) &= \int \frac{dq}{q}W{(qR)}^2 \\
    &= \frac{16}{81}\int \frac{dq}{q} {(qR)}^4 \mathcal{P_R}(q)W{(qR)}^2,
\end{align}

对应$k$模的原初黑洞对暗物质的贡献为
\begin{align*}
    f_{PBH}(M) &\equiv \frac{\Omega_{PBH}(M)}{\Omega_c} =
    \frac{\rho_{PBH}(M)}{\rho_m}\mid_{eq}\frac{\Omega_m h^2}{\Omega_c h^2}
    \\
    &= \frac{\beta(M)}{8\times
    10^{-16}}{\left(\frac{\gamma}{0.2}\right)}^{3/2}{\left(\frac{g(T_M)}{106.75}\right)}^{-1/4}{\left(\frac{M}{10^{18}\ g}\right)}^{-1/2}
\end{align*}

其中暗物质的丰度取为$\Omega_c h^2 \simeq 0.12$\citep{aghanim2018planck}。当前所有原初黑洞的丰度由积分式给出;
\begin{align}
    \label{eq:abundence_pbh}
    \Omega_{PBH} = \int \frac{dM}{M}\Omega_{PBH}(M),
\end{align}

公式{(\ref{eq:mass_fraction_pbh})}说明原初黑洞的质量分数对临界塌缩密度$\delta_c$非常敏感。在辐射为主时期,近期多数在这方面的研究文章\citep{musco2005computations,musco2009primordial,musco2013primordial,harada2013threshold}建议$\delta_c$取值大约为$0.45$。此时若希望原初黑洞能在$\mathcal{O}(1)$量级成为暗物质组成成分,那么曲率扰动的原初功率谱需要增大到大约为$\mathcal{P_R}\simeq
10^{-2}$。而在CMB尺度上,扰动大约为$\mathcal{P_R}\simeq
10^{-9}$。因此我们需要一个机制,使得对应某个尺度范围(相应的某个原初黑洞的质量区间)的曲率扰动在离开视界时产生一个峰。

把原初黑洞的质量和e-folding数联系起来,能使我们对其形成于哪个时期有一个相对清晰的概念。为了做到这一点,需要假设在暴涨时期,哈勃“常数”近似为常数。所以$k$模离开视界时相对于某个基准$k_\star$已经膨胀了大约
\begin{align}
    \label{eq:delta_efolding}
    \Delta N^{\star}_e = \log \frac{a_k}{a_\star} = \log\frac{a_k
    H_I}{k_\star} = \log \frac{a_f H_f}{k_\star},
\end{align}
仍然假设对应熵和能量密度的有效自由度数相等,于是可以得到公式\citep{motohashi2017primordial}:
\begin{align}
    \Delta N^{\star}_e =
    -\frac{1}{2}\log\frac{M}{M_\odot}+\frac{1}{2}\log\gamma
    +\frac{1}{12}\log\frac{g(T)}{106.75}+\frac{1}{2}\log\frac{4.4\times10^{24}\Omega_r
    H^2_0}{k^2_{\star}}, 
\end{align}
其中辐射密度为$\Omega_r h^2=4.18\times10^{-5}$,哈勃常数为$H_0\simeq
0.0007\text{Mpc}^{-1}$。若选择基准为$k_\star =
  0.05\text{Mpc}^{-1}$,参考Planck组的数据,上式可以变为
\begin{align}
    \Delta N^{\star}_e = 18.37-\frac{1}{2}\log\frac{M}{M_\odot}.
\end{align}


