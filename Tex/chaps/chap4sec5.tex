\section{小结}

这一章我们在第一个工作的基础上,在超引力框架中构造了一个双拐点暴胀模型。其中
一个拐点产生与CMB观测一致的功率谱。另一个拐点使原初功率谱在小尺度上产生一个峰值
,这个峰不仅能使扰动在辐射为主时期重新进入视界时产生原初黑洞,同时也能让诱导
引力波的信号在该尺度附近得到增强。由于在拐点附近势能极端平坦,因而通常的慢滚近似
不再适用,只能通过数值算法得到曲率扰动的功率谱。另一方面,我们在共形牛顿规范下
求得诱导引力波满足的演化方程,并且得到了诱导引力波的功率谱$\mathcal{P}_{h}$和曲率扰动的
功率谱$\mathcal{P}_{\mathcal{R}}$的关系,最终获得了诱导引力波能谱的表达式。
进行计算之后,发现诱导引力波的信号存在一个峰,且峰的高度在LISA与Taiji的期望
灵敏度曲线之上,意味着模型预言的诱导引力波信号有望被LISA和Taiji的实验观测到。
