\section{构造拐点暴胀模型}
这里我们讨论如何在超引力中构建一个拐点暴胀模型。基于超引力构建的大多数旧暴胀模型\citep{freedman1989progress,deser1976consistent,wess1992supersymmetry}都遇到了$\eta$问题\citep{yamaguchi2011supergravity}:F-term势能正比于指数因子$e^{|\phi|^2}$,使得慢滚参数$\eta$不满足慢滚条件。有多种方法可以$\eta$问题\citep{stewart1995inflation,linde1994hybrid,linde1997hybrid,panagiotakopoulos1997hybrid}。其中一种办法是将平移对称性引入K\"ahler势中$K(\phi-\phi^\dagger)$,这样暴胀子不会出现在K\"ahler势中,消失的指数因子保证了势能的平坦性。这种情形下,能够得到幂指数势能$V(\phi)\propto
\phi^n$的混沌暴胀。

有作者\citep{takahashi2010linear,nakayama2010running,kasuya2014flat}将手征场$\phi$的类Nambu-Goldstone平移对称性推广到了一般形式$\phi^n\rightarrow
\phi^n+C$,使得K\"ahler势是$\left(\phi^n-\phi^\dagger\right)$的函数。
为了获得合理的势能形式,应当在K\"{a}hler势和超势中分别加入一个小的shift对称性破缺项$\kappa|\phi|^2$和$\lambda\phi^m
X$,此处有$\kappa,\lambda\ll
1$以及外场$X$。这样可以获得幂指数为分数的标量势能$V(\phi)\propto
\phi^{m/n}$,标量谱指标$n_s$以及张标比$r$近似为
$n_s=1-(1+\frac{m}{n})\frac{1}{N}$和$\frac{8m}{n}\frac{1}{N}$\citep{nakayama2010running}。
由于模型中动能项的系数是场依赖的:$(\kappa+n^2|\phi|^{2n-2})\partial_{\mu}\phi\partial^{\mu}\phi^{\dag}$,这就是所谓的跑动动能项暴胀。

在最小超对称标准模型的框架下,通过规范不变的平坦方向\textit{udd}或\textit{LLe}首次构造出了拐点暴胀模型\citep{allahverdi2006gauge},并得到了发展\citep{allahverdi2007term,enqvist2010inflection,hotchkiss2012observable,chatterjee2015bound}。
在\citep{gao2015inflection}中,作者用单手征超场构造出了一个拐点暴胀模型,与Planck
2015年的结果相一致。在暴胀结束后,该模型进入一个非超对称德-西特真空,能够用于解释宇宙近期的加速膨胀。

\subsection{带跑动动能项的多项式超势}
对于慢滚暴胀模型,习惯假设暴胀场是弱耦合场,其拉氏量中的动能项在暴胀期间及暴胀
结束后不会发生大的改变。又为了暴胀能够产生,通常利用对称性产生平坦的势能。如果
不利用对称性,通常需要精细调节参数来获得平坦的势能,且需要挑选能够使暴胀发生的
初始位置。后来,有人提出了一类新的暴胀模型,丢弃了动能项的变化被忽略的假设。在
整个暴胀期间,动能项的变化不能被忽略,在某些情况下,这种变化甚至能明显地影响到
暴胀场的动力学。这养的模型被称为\textbf{跑动动能暴胀}模型。


K\"ahler势在手征超场$\phi$的广义平移对称性变换下保持不变:
\begin{equation}
    \phi^n \rightarrow \phi^n + C,
\end{equation}

因此,它必然是$i\chi \equiv (\phi^n - \phi^{n\dagger})$的函数:
\begin{equation}
    K = \sum_{l=1}\frac{c_l}{l}{\left(\phi^n-\phi^{n\dagger}\right)}^l
    =
    i c_1(\phi^n-\phi^{n\dagger})-\frac{1}{2}{\left(\phi^n-\phi^{n\dagger}\right)}^2+\cdots.
\end{equation}

系数$c_l$是与$1$同量级的常数。在暴胀期间,$\chi$一直稳定在极小值处,所以$l\ge
3$的项对动能的贡献可以忽略\citep{takahashi2010linear}。对$l\le2$的项,$\chi$稳定在$\chi_{\text{min}}\simeq
c_1$。

为了使暴胀模型可行,需要在K\"ahler势中添加一个微小的平移对称性破缺项:
\begin{equation}
    \Delta{K} = \kappa \left|\phi\right|^2,
\end{equation}
系数满足条件$0< \kappa \ll 1$。

与\citep{nakayama2010running}不同,我们假设超势满足一个更一般的形式\citep{nakayama2013polynomial,kawasaki2000natural,kawasaki2001natural,kallosh2010new,kallosh2011general}
\begin{equation}
    W = X\left(\sum_{m=0}\lambda_m\phi^m\right)+W_0.
\end{equation}
这里$\lambda_m\ll
1$,$X$是一个额外的手征超场,并且稳定在$X=0$。$\left|W_0\right|\simeq
m_{3/2}$且$m_{3/2}$为引力微子的质量。按照通常的假设,在暴胀过程中$m_{3/2}$远远小于哈勃参数,所以在讨论暴胀动力学的时候可以被丢弃。

所以最终的K\"ahler势为
\begin{equation}
    K = \kappa |\phi|^2 + c_1(\phi^n-\phi^{n\dagger}) -
    \frac{1}{2}{(\phi^n-\phi^{n\dagger})}^2 + |X|^2\cdots.
\end{equation}

超势$W$和K\"ahler势共同决定了标量场$\phi$的有效拉式量
\begin{equation}
\begin{split}\label{eq:lagrangian}
L &= -K^{\phi\phi^\dagger} \partial_\mu\phi\partial^\mu\phi^\dagger
- V \\
  & = (\kappa + n^2|\phi|^{(2n-2)})
  \partial_\mu \phi \partial^\mu \phi^\dagger - V,
\end{split}
\end{equation}
暴胀场$\phi$的标量势能为
\begin{equation}
\begin{split}
V &= e^K \left[
    D_\phi W{(K^{-1})}^{\phi\phi^\dagger}{(D_\phi W)}^{\star} - 3|W|^2 
    \right]\\
  &=
  e^{\kappa|\phi|^2+c_1(\phi^n-\phi^{n\dagger})-\frac{1}{2}{(\phi^n-\phi^{n\dagger})}^2}{(\sum_m
  \lambda_m|\phi|^m)}^2,
\end{split}
\end{equation}
其中
\begin{equation}
    D_\phi W = \partial_\phi W + (\partial_\phi K) W,
\end{equation}
以及
\begin{equation}
    K^{\phi\phi^\dagger} = \frac{\partial^2 K}{\partial\phi
    \partial\phi^\dagger}
\end{equation}

当${(\kappa/n^2)}^{1/(2n-2)}\ll |\phi| \ll
\kappa^{-1/2}$时,$\kappa$项可以忽略,所以若定义$\hat{\phi} \equiv
\phi^n$,则意味着$\hat{\phi}$在类Nambu-Goldstone平移变换下不变,拉式量
(\ref{eq:lagrangian})近似为
\begin{equation}
\begin{split}
    L &= \partial_\mu \hat{\phi}\partial^\mu \hat{\phi}^\dagger -
    e^{-\frac{|c_1|^2}{2}}{(\sum_m \lambda_m|\hat{\phi}|^{m/n})}^2 \\
    &= \partial_\mu \hat{\phi} \partial^\mu \hat{\phi}^\dagger - 
    {(\sum_m \hat{\lambda_m}|\hat{\phi}|^{m/n})}^2,
\end{split}
\end{equation}
其中$i\chi \equiv (\phi^n - \phi^{n\dagger})=i c_1$,$\hat{\lambda}_m\equiv
e^{-\frac{|c_1|^2}{4}}\lambda_m$。

由于平移对称性,$\hat{\phi}$的实部不会出现在K\"ahler势中,所以势能在实方向上相当地平坦使得其成为暴胀子候选者。
虚部获得了比哈勃能标更重地质量,并且在暴胀过程中稳定在极小值处。因此,设$\hat{\phi}\equiv
(\varphi+i \chi)/\sqrt{2}=\varphi/\sqrt{2}+i
c_1/2$并得到$\varphi$地标量势能
\begin{equation}\label{eq:varphi_scalar_potential}
    V = {\left[\sum_{m=0}\hat{\lambda}_m
    {\left(\frac{\varphi}{\sqrt{2}}\right)}^{\frac{m}{n}}\right]}^2.
\end{equation}

\subsection{建立拐点暴胀模型}
为了创建一个拐点暴胀模型,我们采用如下形式的超势
\begin{equation}\label{eq:superpotential}
    W = X(\lambda_p\phi^p + \lambda_q e^{i\theta}\phi^q),
\end{equation}
其中幂指数$q > p \le
1$且为整数,系数$\lambda_p$、$\lambda_q$为正实数且不失一般性,$\theta$为第二项的相位。

将 (\ref{eq:superpotential})代入势能函数
(\ref{eq:varphi_scalar_potential})得到
\begin{equation}
\begin{split}
    V &= {\left[ \hat{\lambda}_p {\left(\frac{\varphi}{\sqrt{2}}\right)}^{\frac{p}{n}}
    + \hat{\lambda}_q e^{i\theta}{\left(\frac{\varphi}{\sqrt{2}}\right)}^{\frac{q}{n}}\right]}^2 \\
    &= \hat{\lambda}_p^2 {\left(\frac{\varphi}{\sqrt{2}}\right)}^{\frac{2p}{n}}
    \left[1 + 2\xi\cos\theta {\left(\frac{\varphi}{\sqrt{2}}\right)}^{\frac{q-p}{n}}
    + \xi^2 {\left(\frac{\varphi}{\sqrt{2}}\right)}^{\frac{2(q-p)}{n}}\right],
\end{split}
\end{equation}
其中$\hat{\lambda}_{p,q}\equiv
e^{-\frac{|c_1|^2}{4}}\lambda_{p,q}$和$\xi=|\hat{\lambda}_q/\hat{\lambda}_p|=|\lambda_q/\lambda_p|$。
% 势能曲线参见图\ref{fig:varphi_scalar_potential}。

% \begin{figure}
%     \centering
%     \includegraphics{}
%     \caption{暴胀势能$V(\phi)$,其中$\cos\theta=-1$(虚线),$\cos\theta=-2\sqrt{pq}/(p+q)$(实线),$\cos\theta
%     > -2\sqrt{pq}/(p+q)$(点线)}\label{fig:varphi_scalar_potential}
% \end{figure}
