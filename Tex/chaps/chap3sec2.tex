\section{跑动能项暴胀模型}
其次是\textbf{跑动动能暴胀}模型,是指场的质量在暴胀期间会随场的变化而变的模型
。对于慢滚暴胀模型,习惯假设暴胀场是弱耦合场,其拉氏量中的动能项在暴胀期间及暴
胀结束后不会发生大的改变。又为了暴胀能够产生,通常利用对称性产生平坦的势能。如
果不利用对称性,通常需要精细调节参数来获得平坦的势能,且需要挑选能够使暴胀发生
的初始位置。后来,有人提出了一类新的暴胀模型,丢弃了动能项的变化被忽略的假设。
在整个暴胀期间,动能项的变化不能被忽略,在某些情况下,这种变化甚至能明显地影响
到暴胀场的动力学。这样的模型被称为\textbf{跑动动能暴胀}模型。

\textbf{出发点} 

假设实标量场$\phi$有如下形式的拉氏量,
\begin{equation}
  \mathcal{L} =
  \frac{1}{2}f(\phi)\partial^{\mu}\phi\partial_{\mu}\phi-V(\phi),
\end{equation}
假设场$\phi$在某个区间时,$f(\phi)$的行为能够近似为$f(\phi)\approx
\phi^{2n-2}$,$n$为给定的整数。则对场$\phi$作正则归一化之后,$\hat{\phi}\equiv
\phi^{n}/n$,势能函数将会变得更加平坦。例如,当$n=2$时,二次(四次)方势能函数,$V(\phi)\propto
\phi^2(\phi^{4})$,将变成线性(二次)势能$V(\hat{\phi})\propto
\hat{\phi}(\hat{\phi}^2)$。因此暴胀场的动力学非常依赖其动能项。

再举一个例子,
\begin{equation}
  f(\phi) = \kappa + \phi^2. 
\end{equation}
其中$0 < \kappa \ll 1$。因此当$\phi \gtrsim \sqrt{\kappa}$时,$f(\phi)
\approx \phi^2$。当$\phi \ll \sqrt{\kappa}$时,$f(\phi)\approx
\sqrt{\kappa}$。因此,标量场$\phi$将按如下方式进行正则归一化
\begin{equation}
  \hat{\phi} = 
  \begin{cases}
    \frac{\phi^2}{2}, \qquad & \phi \gg \sqrt{\kappa} \\
    \sqrt{\kappa},\qquad & \phi \ll \sqrt{\kappa}
  \end{cases},
\end{equation}
故而当$V(\phi)=m^2\phi^2/ 2$时,正则场的势能函数为
\begin{equation}
  V(\hat{\phi}) \simeq 
  \begin{cases}
    m^2\hat{\phi},\qquad & \hat{\phi} \gg \kappa \\
    \frac{m^2}{2\kappa} \hat{\phi}^2,\qquad & \hat{\phi} \ll \kappa
  \end{cases}.
\end{equation}
从这个例子可以发现,从一个形式简单的拉氏量出发,经过正则归一化,能够轻松实现势能
函数从线性形式到二次函数的转变。
原则上,我们可以把正则场$\hat{\phi}$满足的拉氏量作为出发点,得到的暴胀模型拥有
相同的动力学。然而,这种方式难以从理论上解释为何暴胀场的势能取这样一种特定的函数
形式。


