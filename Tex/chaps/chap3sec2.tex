\section{跑动动能项暴涨}
对于慢滚暴涨模型,习惯假设暴涨场是弱耦合场,其拉氏量中的动能项在暴涨期间及暴
涨结束后不会发生大的改变。又为了暴涨能够产生,通常利用对称性产生平坦的暴涨势。如
果不利用对称性,通常需要精细调节参数来获得平坦的暴涨势,且需要挑选能够使暴涨发生
的初始位置。后来,有人提出了一类新的暴涨模型,丢弃了动能项的变化被忽略的假设。
在整个暴涨期间,动能项的变化不能被忽略,在某些情况下,这种变化甚至能明显地影响
到暴涨场的动力学。这样的模型被称为\textbf{跑动动能暴涨}模型。

假设实标量场$\phi$有如下形式的拉氏量,
\begin{equation}
  \mathcal{L} =
  \frac{1}{2}f(\phi)\partial^{\mu}\phi\partial_{\mu}\phi-V(\phi),
\end{equation}
其中标量场$\phi$为暴涨场,并且假设在势函数取最小值处,拉氏量是正则归一化的:
\begin{equation}
  f(\phi_{\min})=1. 
\end{equation}
这并不意味着在暴涨期间$f(\phi)$需要始终接近$1$,尤其是当暴涨场$\phi$处在
大统一能标或是普朗克能标附近时。我们假设场$\phi$在某个区间时,
$f(\phi)$的行为能够近似为$f(\phi)\approx
\phi^{2n-2}$,$n$为给定的整数。则对场$\phi$作正则归一化之后,$\hat{\phi}\equiv
\phi^{n}/n$,暴涨势将会变得更加平坦。例如,当$n=2$时,二次(四次)方暴涨势,$V(\phi)\propto
\phi^2(\phi^{4})$,将变成线性(二次)暴涨势$V(\hat{\phi})\propto
\hat{\phi}(\hat{\phi}^2)$。因此暴涨场的动力学非常依赖其动能项。

再举一个例子,
\begin{equation}
  f(\phi) = \kappa + \phi^2. 
\end{equation}
其中$0 < \kappa \ll 1$。因此当$\phi \gtrsim \sqrt{\kappa}$时,$f(\phi)
\approx \phi^2$。当$\phi \ll \sqrt{\kappa}$时,$f(\phi)\approx
\sqrt{\kappa}$。因此,标量场$\phi$将按如下方式进行正则归一化
\begin{equation}
  \hat{\phi} = 
  \begin{cases}
    \frac{\phi^2}{2}, \qquad & \phi \gg \sqrt{\kappa} \\
    \sqrt{\kappa}\phi,\qquad & \phi \ll \sqrt{\kappa}
  \end{cases}
\end{equation}
故而当$V(\phi)=m^2\phi^2/ 2$时,正则场的暴涨势为
\begin{equation}
  V(\hat{\phi}) \simeq 
  \begin{cases}
    m^2\hat{\phi},\qquad & \hat{\phi} \gg \kappa \\
    \frac{m^2}{2\kappa} \hat{\phi}^2,\qquad & \hat{\phi} \ll \kappa
  \end{cases}
\end{equation}
从这个例子可以发现,从一个形式简单的拉氏量出发,经过正则归一化,能够轻松实现暴涨势从线性形式到二次函数的转变。
原则上,我们可以把正则场$\hat{\phi}$满足的拉氏量作为出发点,得到的暴涨模型拥有
相同的动力学。然而,这种方式难以从理论上解释为何暴涨场的暴涨势取这样一种特定的函数
形式。


