\section{拐点暴涨模型}
拐点暴涨通常指这样的暴涨模型,存在场值$\phi=\phi_0$(拐点)使得暴涨势在该处的二阶导数为零
\begin{equation}
    V_{,\phi \phi}(\phi_0) = 0,
\end{equation}

暴涨势的一般形式为

\begin{equation}
    \label{eq:inf_potential}
    V (\phi)=V_0+a (\phi-\phi_0)+\frac{c}{6}{\left(\phi-\phi_0\right)}^3,\\
\end{equation}
\begin{equation}
    V_0\equiv V (\phi_0),\ a\equiv V_{,\phi} (\phi_0),\ c\equiv V_{,\phi \phi \phi}
    (\phi_0),
\end{equation}

高阶项 $(n \geq 4)$ 被截断。相应的慢滚参数为
\begin{align}
    \label{eq:psrp}
    \epsilon &= \frac{1}{2}\lrp{\frac{V_{,\phi}}{V(\phi)}}^2,\\
    \eta&=\frac{V_{,\phi \phi}}{V(\phi)}, \\
    \xi^2&=\frac{V_{,\phi} V_{,\phi \phi \phi}}{V^2(\phi)}.
\end{align}
接下来通过WMAP、Planck等实验观测到的标量扰动大小$\mathcal{P}_{\mathcal{R}}$和标量谱指标$n_s$来确定系数$a$和$c$。
令暴涨结束时的场值为$\phi_e$,那时有$\epsilon\sim
1$。则对应$\phi\sim\phi_e$的e-folding数为
\begin{align}
    \label{eq:e-folding}
    \mathcal{N} &= \frac{V_0}{M^2_p}\sqrt{\frac{2}{ac}}\lbrack
    F_0(\phi_e)-F_0(\phi)\rbrack, \\
    F_0(z) &= \cot^{-1}\left(\sqrt{\frac{c}{2a}}(z-\phi_0)\right).
\end{align}

如果将慢滚参数在极值处的平方根定义为 $X$, $\mathcal{N}$ 中涉及 $\phi$
的部分定义为 $Y$,那么可以将各慢滚参数及可观测量用统一用$X,Y$进行描述。

\begin{align}
    \epsilon &=
    \frac{2V_0^2}{c^21^6\mathcal{N}^4} {\left(\frac{Y}{X} \right)}^4, \\
    \eta &=
    -\frac{2}{\mathcal{N}}\frac{Y}{S}\left(\sqrt{1-X}\cos Y-\sqrt{X}\sin
    Y\right),\\
    \xi^2 &= \frac{2}{\mathcal{N}^2}{\left(\frac{Y}{X}\right)}^2.
\end{align}

在暴涨期间,$\epsilon \ll 1$,导致$X=S\sqrt{\epsilon}\leq\sqrt{\epsilon}\ll
1$,故而有$S\approx \sin Y$。相应的功率谱和标量扰动谱指标为
\begin{align}
  \mathcal{P}_{\mathcal{R}}^{1/2}&\equiv\frac{1}{\sqrt{24\pi^2}}\frac{V_0^{1/2}}{\epsilon^{1/2}1^2}
    \approx\frac{V_0^{1/2}}{2\sqrt{6}\pi 1^2 X}\sin^2Y,\\
    n_s&\equiv1+2\eta-6\epsilon\approx1-\frac{4}{\mathcal{N}}Y\cot Y.
\end{align}
