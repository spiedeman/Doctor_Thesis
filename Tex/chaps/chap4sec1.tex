\section{双拐点暴胀模型}
拐点暴胀通常指这样的暴胀模型,存在场值$\phi=\phi_0$(拐点)使得势能在该处的二阶导数为零
\begin{equation}
    V^{\dprime}(\phi_0) = 0
\end{equation}

势能的一般形式为

\begin{equation}
    \label{eq:inf_potential}
    V (\phi)=V_0+a (\phi-\phi_0)+\frac{c}{6}{\left(\phi-\phi_0\right)}^3\\
\end{equation}
\begin{equation}
    V_0\equiv V (\phi_0),\ a\equiv V' (\phi_0),\ c\equiv V^{\tprime} (\phi_0) 
\end{equation}

高阶项 $(n \geq 4)$ 被截断。相应的慢滚参数为
\begin{equation}
    \label{eq:psrp}
    \epsilon = \frac{M_p}{2}\left(\frac{V'}{V}^2\right),\ \eta=M^2_p
    \left(\frac{V^{\dprime}}{V}\right),\ \xi^2=M^4_p\left(\frac{V'V^{\tprime}}{V^2}\right)
\end{equation}

接下来通过WMAP、Planck等实验观测到的标量扰动大小$\mathcal{P}_R$和标量谱指标$n_s$来确定系数a和c。
令暴胀结束时的场值为$\phi_e$,那时有$\epsilon\sim
1$。则对应$\phi\sim\phi_e$的e-folding数为
\begin{align}
    \label{eq:e-folding}
    \mathcal{N} &= \frac{V_0}{M^2_p}\sqrt{\frac{2}{ac}}\lbrack
    F_0(\phi_e)-F_0(\phi)\rbrack \\
    F_0(z) &= \cot^{-1}\left(\sqrt{\frac{c}{2a}}(z-\phi_0)\right)
\end{align}

如果将慢滚参数在极值处的平方根定义为 $X$, $\mathcal{N}$ 中涉及 $\phi$
的部分定义为 $Y$,那么可以将各慢滚参数及可观测量用统一用$X,Y$进行描述。

\begin{align}
    \epsilon &=
    \frac{2V_0^2}{c^2M_p^6\mathcal{N}^4} {\left(\frac{Y}{X} \right)}^4 \\
    \eta &=
    -\frac{2}{\mathcal{N}}\frac{Y}{S}\left(\sqrt{1-X}\cos Y-\sqrt{X}\sin
    Y\right)\\
    \xi^2 &= \frac{2}{\mathcal{N}^2}{\left(\frac{Y}{X}\right)}^2
\end{align}

在暴胀期间,$\epsilon \ll 1$,导致$X=S\sqrt{\epsilon}\leq\sqrt{\epsilon}\ll
1$,故而有$S\approx \sin Y$。相应的功率谱,标量谱指标及其跑动为
\begin{align}
    \mathcal{P}^{1/2}&\equiv\frac{1}{\sqrt{24\pi^2}}\frac{V_0^{1/2}}{\epsilon^{1/2}M_p^2}
    \approx\frac{V_0^{1/2}}{2\sqrt{6}\pi M_p^2 X}\sin^2Y\\
    n_s&\equiv1+2\eta-6\epsilon\approx1-\frac{4}{\mathcal{N}}Y\cot Y\\
\end{align}
