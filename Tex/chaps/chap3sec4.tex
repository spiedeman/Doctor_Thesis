\section{小结}

这一章首先介绍了在超引力框架内构造暴涨模型的历史。早期基于超引力的暴涨模型都会
遇到$\eta$问题,通常需要精细条件参数才能形成合理的暴涨过程。然而即便采用参数
调节的方式,在很多情况下也难以做到。后来通过引入平移对称性解决了这个问题。
接着回顾了跑动动能项暴涨模型。该模型的特点是在暴涨期间,拉氏量中动能项的变化
不能被忽略,从而标量场在暴涨过程中和暴涨结束后有不同的标量势。然后介绍了我们的
第一个工作,利用平移对称性以及带跑动动能项的多项式超势在超引力中构造了拐点暴涨模型。
我们的模型预言了与Planck
2015的数据相一致的CMB功率谱,并且对标量功率谱的谱指标$n_{s}$和张标比$r$的预测优于标量势
为$V\propto \varphi^{m
/n}$的原始模型。最后,在暴涨结束后,暴涨场在原点附近来回振荡,通过与希格斯粒子
的耦合逐渐衰变为标准粒子重新加热宇宙。
