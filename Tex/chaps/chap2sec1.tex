\section{暴胀介绍}


- 为何引入暴胀理论


- 暴胀模型基本概念

- 暴胀基本性质

- 具体暴胀模型举例

\subsection{现代宇宙学}
基于当前观测,现代宇宙学普遍采用两条假设作为宇宙学原理:物质分布在宇宙学尺度上
1)均匀的;2)、各向同性。以此为出发点可以导出用于描述可观测宇宙的
Friedmann-Robertson-Walker度规

\begin{equation}\label{eq:frw}
    ds^2=dt^2-a^2(t)\left[\frac{dr^2}{1-Kr^2}+r^2\sin^2\theta d\phi^2\right]
\end{equation}

$a(t)$为尺度因子,$K=-1, 0,
+1$分别代表开放、平坦、闭合宇宙。为了确定宇宙的演化过程,首先求解FRW度规$(\ref{eq:frw})$下的Einstin场方程

\begin{equation}\label{eq:ij_einstein}
    -\frac{2K}{a^2}-\frac{2\dot{a}^2}{a^2}-\frac{\ddot{a}}{a}=-4\phi G(\rho-p)
\end{equation}
\begin{equation}\label{eq:00_einstein}
    \frac{3\ddot{a}}{a}=-4\phi G(3p+\rho)
\end{equation}

上述方程分别来自 Einstein 场方程的 ij分量$(\ref{eq:ij_einstein})$ 和
00分量$(\ref{eq:00_einstein})$。进一步可以导出主导宇宙膨胀的 Friedmann
方程和能量守恒方程

\begin{equation}\label{eq:friedmann_equation}
    H^2+\frac{K}{a^2}=\frac{8\pi G}{3}\rho
\end{equation}
\begin{equation}\label{eq:energy_conservation}
    \dot{\rho}=-\frac{3\dot{a}}{a}(\rho+p)
\end{equation}

定义临界能量密度 $\rho_{cr}=3H^2/(8\pi G)$,以及参数 $\Omega\equiv
\rho/\rho_{cr}$,空间曲率 $K$可以表示成

\begin{equation}\label{eq:flatness}
    \Omega(t) - 1 = \frac{K}{{(Ha)}^2} 
\end{equation}

目前的观测支持 $\Omega(t_0)=1$,即当前的宇宙是平坦的。同时方程
$(\ref{eq:flatness})$ 说明
宇宙越早期越平坦。当回溯到普朗克时间的量级时$\Omega_i-1 \leq
10^{-56}$。如此小的一个误差量级在物理上很不自然,而大爆炸宇宙学无法给出合理的解释。

\subsection{暴胀模型}
暴胀的基本图像是宇宙在极早期经历过一个加速膨胀的过程,之后转为减速膨胀,过渡到大爆炸宇宙模型。

当暴胀场用标量场$\varphi$描述时,能量密度为
\begin{equation}\label{eq:energy_density}
    \rho = \frac{1}{2}\dot{\varphi}^2+V(\varphi)
\end{equation}
以及压强
\begin{equation}\label{eq:pressure}
    p=\frac{1}{2}\dot{\varphi}^2-V(\varphi)
\end{equation}

相应的能量守恒方程$(\ref{eq:energy_conservation})$
以及Friedmann方程$(\ref{eq:friedmann_equation})$改写为
\begin{equation}
    \ddot{\varphi}+3H\dot\varphi+V_{,\varphi}=0 
\end{equation}
\begin{equation}
    H^2=\frac{8\pi G}{3}\left(\frac{1}{2}\dot\varphi^2+V(\varphi)\right)
\end{equation}
约定约化普朗克质量$M_p\equiv\frac{1}{\sqrt{8\pi G}}=1$。消去标量场$\phi$的二阶导数,得到Hamilton-Jacobi方程
    \begin{align}
        \lbrack H'(\varphi)\rbrack^2 - \frac{3}{2}H^2 &=
        -\frac{1}{2}V(\varphi)\label{HJa} \\
        \dot\varphi  &= -2H'(\varphi)\label{HJb}
    \end{align}

    由$(\ref{HJb})$可知$\dot H=-\dot \varphi^2/2\leq 0$,故物理哈勃半径$1/H$随时间增大。同时尺度因子$a(t)$加速增大,故共动哈勃半径$1/(aH)$不断随时间减小。
    方程$(\ref{HJa})$有一个很好的性质,与任意一个方程解的线性扰动都会以指数的速度趋于零,因此很容易进行数值求解。
