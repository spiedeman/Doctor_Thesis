\section{暴涨}
由于共动距离不随时间改变,在共动坐标系下讨论\textbf{(共动)粒子视界}会使视界问
题变得更简单。若标度因子$a(t)$是时间$t$的幂函数,则在当前时刻$t_0$和过去某一时刻$t_1$的因果联系区的共动尺度比为
\begin{align}
  \frac{\chi_{p 0}}{\chi_{p 1}}\sim \frac{t_0}{t_1}\frac{a_1}{a_0}
  \sim \frac{\dot{a}_1}{\dot{a}_0}.
\end{align}

平坦性问题中,给大爆炸宇宙学造成困难的地方来自于比值
\begin{align}
  \label{eq:flatness-of-space}
  \frac{\Omega_0-1}{\Omega_1-1}=
  \frac{{\left(Ha\right)}^2_{1}}{{\left(Ha\right)}^2_{0}}
  \sim {\left(\frac{\dot{a}_1}{\dot{a}_0}\right)}^2.
\end{align}
在这几个问题中,都出现了比值$\dot{a}_1 /
\dot{a}_0$。当强能量条件满足时$\rho+3p>0$,根据加速度方程
$(\ref{eq:accelaration_equation})$标度因子的二阶导数小于零,因此宇宙处于减速
膨胀的状态。故越早期,比值$\dot{a}_1 /\dot{a}_0$越大,问题越严重。如果要求
$\dot{a}_1 /\dot{a}_0\leq 1$,那么只能假设$\dot{a}$经历了先增大、后渐小的过程。
引入的这一段加速膨胀的过程是必要的,至于是否充分取决于实现这一条件的特定模型。
根据加速度方程,普通物质的存在提供了引力使得宇宙的膨胀速度减小。那么为了实现
加速膨胀,必须假设宇宙中存在别的物质,使得引力表现为一种“排斥力”。

我们先给出暴涨的定义:\textbf{暴涨}是当引力表现为一种排斥力使得宇宙处于加速膨胀的一个阶段。

暴涨能够解释大爆炸的起源。因为加速膨胀,原本处于因果联系区内的一个小区域会在短时间内膨胀地非常大。更进一步,
暴涨甚至能从一个均匀的小区域中产生整个可观测宇宙,即便这一区域外相当非均匀。原因在于,在一个加速膨胀的宇宙中,
总是存在有限大的事件视界
\begin{align}
    d_{e}(t) &= a(t)\int_t^{t_{\text{max}}} \frac{dt}{a}
    =a(t)\int_{a(t)}^{a_{\text{max}}} \frac{da}{\dot{a}a}\notag \\
    &= \sqrt{3}a(t)\int_{a(t)}^{a_{\text{max}}}
    \frac{da}{a\sqrt{a^{-(1+3w)}-3K}}.
\end{align}
因为${\left(1+3w\right)}<0$,即使$a_{\text{max}}\rightarrow
\infty$,$d_{e}(t)$也会收敛到有限值。如此一来,以$
p$为中心,$d_{e}(t)$为半径的区域$V_1(t)=\{q\ |\
|q-p|<d_{e}(t)\}$其未来的演化永远不会受到来自于半径为$2d_{e}(t)$的同心区域$V_2(t)$外的影响。能对内部产生作用的只有半径在$d_{e}(t)$到$2d_{e}(t)$的部分。假设在$t=t_{i}$时$V_2(t_{i})$内部的物质分布均匀且各向同性,那么到暴涨结束时,$V_1(t_{f})$内始终保持均匀和各向同性。$V_1(t)$区域的物理尺度在暴涨结束时为
\begin{align}
  d_{e}(t_{f})=d_e(t_{i}) \frac{a_{f}}{a_{i}}.
\end{align}
在加速膨胀宇宙中,粒子视界可以作如下近似
\begin{align}
  d_{p}(t) =a(t)\int_{t_{i}}^{t} \frac{dt}{a}=a(t)\int_{a_{i}}^{a}
  \frac{da}{\dot{a}a}\sim \frac{a(t)}{a_{i}}d_{e}(t_i). 
\end{align}
因为积分的主要贡献来自于$a\sim a_{i}$的部分。在暴涨结束时,$d_{p}(t)\sim
d_{e}(t)$,说明均匀区域的大小与粒子视界的尺度相当。不同于一个均匀的宇宙包含很多不同的因果关联区,我们可以从一个
很小的均匀的因果关联区出发,通过暴涨使得它的大小以爆炸似的方式变大,同时保持了均匀性。

下一个问题是\textbf{均匀性}的条件是否可以放开,即不要求暴涨开始时因果关联区内部物质分布的均匀性。那么我们
从一个高度非均匀的因果关联区出发,计算一下暴涨结束时的扰动大小。假设初始的相对
能量密度扰动在$\sim
H_{i}^{-1}$的尺度上为$\mathcal{O}(1)$的量级,
\begin{align}
  {\left(\frac{\delta \rho}{\rho}\right)}_{t_{i}}
  \sim \frac{1}{\rho} \frac{|\nabla\rho|}{a_{i}}H_{i}^{-1}
  =\frac{|\nabla\rho|}{\rho}\frac{1}{\dot{a}_{i}}
  \sim \mathcal{O}(1),
\end{align}
这里$\nabla$是在共动坐标系中对空间的导数。当$t\gg
t_{i}$时,哈勃半径$H^{-1}(t)$内的能量密度扰动近似为
\begin{align}
  \label{eq:relative-energy-density-perturbation}
  {\left(\frac{\delta\rho}{\rho}\right)}_{t}
  \sim \frac{1}{\rho} \frac{|\nabla\rho|}{a(t)}H^{-1}(t)
  \sim \mathcal{O}(1) \frac{\dot{a}_{i}}{\dot{a}(t)}.
\end{align}

这里假设了$|\nabla\rho|/\rho$不会在暴涨期间产生明显变化。这一假设的合理性可以来自于对标量扰动在大于$H^{-1}$的尺度上的行为的分析。
$(\ref{eq:relative-energy-density-perturbation})$说明当宇宙处于加速膨胀过程中时,在半径为$H^{-1}$的区域内,
相对能量密度扰动会变得越来越小。因而宇宙变得越来越平坦。

总结起来,如今我们观测到的均匀且各向同性的宇宙,可以有一个非均匀的起源,只要存在暴涨就可以将非均匀性给抹除。

同时,暴涨也能解决平坦性问题。从$(\ref{eq:relative-energy-density-perturbation})$可以发现,如果要求避免大的密度扰动重新进入视界($\sim
H_0^{-1}$)使得宇宙的均匀性假设被破坏,那么必须假设暴涨开始时候的膨胀率远小于今天的膨胀率,即$\dot{a}_{i}/\dot{a}_0\ll
1$。
再精确一点,可以根据CMB的观测要求在当今的视界尺度上,能量密度扰动的方差不应当超过$10^{-5}$。
只要满足条件$\dot{a}_{i}/\dot{a}_0\ll
10^{-5}$,方程$(\ref{eq:relative-energy-density-perturbation})$即可被满足,暴涨时的非均匀性足够被抹除。
$(\ref{eq:flatness-of-space})$可以改写为
\begin{align}
  \Omega_0=1+{\left(\Omega_{i}-1\right)}{\left(\frac{\dot{a}_{i}}{\dot{a}_0}\right)}^2.
\end{align}
可以看出只要满足条件$|\Omega_{i}-1|\sim\mathcal{O}(1)$,就能使得
\begin{align}
  \Omega_0=1.
\end{align}
在相当高的精度上成立。还可以从另一个角度来理解,在减速膨胀的宇宙中,只当$t\rightarrow
0$时会有$\Omega(t)\rightarrow
1$。而在加速膨胀的宇宙中,是当$t\rightarrow\infty$时才有$\Omega(t)\rightarrow
1$。因而只要在减速膨胀阶段之前存在一个加速膨胀阶段,困扰我们的精细调节问题便不复存在。



一般我们通过标量场来实现一个强能量条件$(\rho+3p >
0)$被破坏的模型,相应的场被称为暴涨场。带有暴涨势$V(\varphi)$的标量场$\varphi$的能动量张量为
\begin{align}
  \label{eq:scalar-energy-momentum-tensor}
  T^{\alpha}_{\ \beta} = \varphi^{,\alpha}\varphi_{,\beta}-
  \frac{1}{2}{\left(\varphi^{,\gamma}\varphi_{,\gamma}-V(\varphi)\right)}\delta^{\alpha}_{\
  \beta}.
\end{align}
根据$T^{\alpha}_{\ \beta;\alpha}=0$可得
\begin{align}
  \varphi^{;\alpha}_{\ ;\alpha}+V_{,\varphi}
  =0,
\end{align}
能动量张量$T^{\alpha}_{\
\beta}$可以改写为理想流体的能动量张量形式,
\begin{align}
  \label{eq:perfect-fluid-energy-momentum-tensor}
  T^{\alpha}_{\ \beta}=(\rho+p)u^{\alpha}u_{\beta}-p\delta^{\alpha}_{\
  \beta}.
\end{align}
若$\varphi^{,\gamma}\varphi_{,\gamma}>0$,且
\begin{align}
  \rho &= \frac{1}{2}\varphi^{,\gamma}\varphi_{,\gamma}+V(\varphi),\\
  p&= \frac{1}{2}\varphi^{,\gamma}_{,\gamma}-V(\varphi), \\
  u^{\alpha}&= \varphi^{,\alpha}/
  \sqrt{\varphi^{,\gamma}\varphi_{,\gamma}},
\end{align}
特别当$\varphi$场是均匀场,即$\partial\varphi /\partial x^{i}=0$时场的能量密度为
\begin{equation}\label{eq:energy_density}
  \rho = \frac{1}{2}\dot{\varphi}^2+V(\varphi),
\end{equation}
以及压强
\begin{equation}\label{eq:pressure}
  p=\frac{1}{2}\dot{\varphi}^2-V(\varphi).
\end{equation}
相应的连续性方程$(\ref{eq:continuation})$
以及弗里德曼方程$(\ref{eq:1st_friedmann_equation})$改写为
\begin{equation}
  \label{eq:continuation_in_inflation}
  \ddot{\varphi}+3H\dot\varphi+V_{,\varphi}=0,
\end{equation}
\begin{equation}
  \label{eq:1st_friedmann_equation_in_inflation}
  3H^2=\frac{1}{2}\dot\varphi^2+V(\varphi).
\end{equation}
% 约定约化普朗克质量$M_p\equiv\frac{1}{\sqrt{8\pi G}}=1$。
消去标量场$\varphi$的二阶导数,得到Hamilton-Jacobi方程
\begin{align}
  \lbrack H_{,\varphi} \rbrack^2 - H^2 &=
  -V(\varphi)\label{HJa}, \\
  \dot\varphi  &= -2H_{,\varphi} \label{HJb}.
\end{align}

由$(\ref{HJb})$可知$\dot H=-\dot \varphi^2 /2\leq 0$,故物理哈勃半径$1/H$
随时间增大。同时标度因子$a(t)$加速增大,故共动哈勃半径$1/(aH)$不断随时间减小。
方程$(\ref{HJa})$有一个很好的性质,即任意一个方程解的线性扰动都会以指数的速度趋于零,因此很容易进行数值求解。

因为$\rho+3p=2{(\dot{\varphi}^2-V(\varphi))}$,所以只有当$\dot{\varphi}^2<V(\varphi)$时,宇宙才能加速膨胀。
而在常见的慢滚暴涨模型中,假设动能项远小于势能项,$\dot{\varphi}\ll
V(\varphi)$,并且摩擦项$3H\dot{\varphi}$足够大,使得加速度项$\ddot{\varphi}$迅速减小直至可被忽略。
基于慢滚暴涨模型的上述两个假设,方程$(\ref{eq:continuation_in_inflation})$及$(\ref{eq:1st_friedmann_equation_in_inflation})$简化为
\begin{align}
  \label{eq:friedmann_equation_in_slow_roll_inflation}
  3H\dot{\varphi} &= -V_{,\varphi} , \\
  3H^2&=V(\varphi).
\end{align}
为了更好地刻画与描述慢滚暴涨过程,通常借助慢滚参数。慢滚参数有两种定义方式,基于暴涨势和哈勃参数的慢滚参数(SRP)。
分别以下标V和H表示,定义势能慢滚参数
\begin{align}
  \label{eq:PSRA_epsilon}
  \epsilon_V(\varphi) &= \frac{1}{2}
  {\left(\frac{V_{,\varphi}}{V(\varphi)}\right)}^2,
  \\
  \label{eq:PSRA_eta}
  \eta_V(\varphi) &= \frac{V_{,\varphi \varphi}}{V(\varphi)}.
\end{align}
及哈勃慢滚参数
\begin{align}
  \label{eq:HSRA_epsilon}
  \epsilon_H(\varphi) &= 2
  {\left(\frac{H_{,\varphi}}{H(\varphi)}\right)}^2=-\frac{\dot{H}}{H^2}, \\
  \label{eq:HSRA_eta}
  \eta_H(\varphi) &= 2 \frac{H_{,\varphi \varphi}}{H(\varphi)}=-3
  \frac{\ddot{\varphi}}{3H\dot{\varphi}}.
\end{align}
相比$\epsilon_V$和$\eta_V$,采用$\epsilon_H$和$\eta_H$描述暴涨的优势在于
\begin{itemize}
  \item $\epsilon_H\ll 1$是动能项相比于势能项可被忽略的充要条件。\\
  \item $\lvert \eta_H\rvert \ll
    1$是加速度项相比于摩擦项可被忽略,不需要额外假设存在吸引子解的充要条件。
    因此哈勃SRP中包含了所有必要的动力学信息。 \\
  \item 暴涨发生的充要条件为
    \begin{equation}
      \ddot{a} > 0 \Longleftrightarrow \epsilon_H < 1. 
    \end{equation}
\end{itemize}



% 能够退出暴涨是一个暴涨模型有效的必要条件之一。假设$H^2$比$\dot{H}$变化的更快,则暴涨持续的时间根据$(\ref{eq:accelaration_equation})$可得
% \begin{align}
%   \label{eq:time_for_inflation}
%   t_f \sim H_i/\lvert H_i\rvert,
% \end{align}
% 为了保证不出现平坦性疑难,暴涨结束时需满足条件
% \begin{align}
%    \frac{a_{f}}{a_{i}} > 10^{33}\frac{H_{i}}{H_{f}},
% \end{align}
% 假设$\lvert H_{i}\rvert \ll
% H^2_{i}$且哈勃参数不随时间变化,则由上式可得估计
% \begin{align}
%   \frac{\lvert\dot{H_i}\rvert}{H^2_i} < \frac{1}{75},
% \end{align}
% 假设$k=0$,利用上式可以得到初始状态方程满足的约束条件
% \begin{align}
%   \frac{\rho_i+p_i}{\rho_i} < 10^{-2}. 
% \end{align}
% 换言之,暴涨开始时对真空物态方程的偏离不超过$1\%$.
